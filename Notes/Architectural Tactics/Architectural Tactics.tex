\documentclass[a4paper,12pt]{article}
\begin{document}

\title{Architectural tactics or strategies - Notes}
\author{Keorapetse Shiko}
\date{09 March 2016}
\maketitle

\cleardoublepage

\section{Architectural Tactics}

	\subsection{Introduction}
	Quality requirements have a significant influence on the software architecture of a system. Architectural tactics are techniques that an 
	architect may use to comprehensively achieve the quality requirements. For each quality requirement listed, potential tactics will be mentioned.

	\subsection{Quality requirements}\label{subsec:quality}
	\begin{itemize}
		\item Performance
		\begin{itemize}
			\item The system is fairly small and simple. Any form or action of enhancing performance may be relatively negligible but a couple of tactics can be implemented. 
			
			\item Tactics
			\begin{itemize}
					\item Spread load across resources
						
						\begin{itemize}
							\item Concurrency can be introduced. Processing different streams of events on different threads and creating additional threads to process different sets of activities can be done by processing requests in parallel. Appropriately allocating threads to resources can maximally exploit concurrency. 
							
						\end{itemize}
					
					\item Resource re-use, caching.
					
						\begin{itemize}
							\item Considering the Readers/Writers problem, multiple readers will have access to view the resource while only one writer can have access to edit the resource. Multiple copies of data can be maintained. Caching is a tactic that replicates data which would be on the same repository to reduce contention.
						\end{itemize}
						
			\end{itemize}		
		\end{itemize}
		
		\cleardoublepage
		
		\item Reliability
		\begin{itemize}
			
			\item Tactics
			\begin{itemize}
				\item Resource locking
				
					\begin{itemize}
						\item A resource or a part/section of the resource will be locked, the correct resource will be returned by making sure that the correct resource is requested to the database and only valid requests will be approved. After the resource is returned, only one user will be able to edit that resource. The other users will be notified about the lock by getting conflicts when they try to pull the resource.
					\end{itemize}
										
				\item Maintain backup
				
					\begin{itemize}
						\item The database will have a back up of all of the information that it stores.
					\end{itemize}
								

			\end{itemize}
						
		\end{itemize}
		\item Scalability
		\begin{itemize}
			
			\item Tactics
			\begin{itemize}

				\item Increasing capacity of communication channels
					
					\begin{itemize}
						\item Users will access the system through a website front-end on a web server maintained by the department. The site will be browser independent in order to increase communication channels. The Android system will also be used as a replacement for the website ans vice versa which allow use of mobile phones or tablets.
					\end{itemize}

			\end{itemize}
				
				
		\cleardoublepage
		
						
		\end{itemize}
		\item Security
		\begin{itemize}
			
			\item Tactics
			\begin{itemize}
				
				\item Authentication
					
					\begin{itemize}
						\item Users will be authenticated via their usernames and passwords by making use of server side validation.
					\end{itemize}
					
				\item Encryption
				
					\begin{itemize}
						 \item Passwords are to be hashed using at least sha256 and should be stored as such within the database along with a salt.
					\end{itemize}				 		
						 
				\item Authorisation
						\begin{itemize}
							\item Users will be grouped by user classes to ensure access control. Users rights to access and modification of data will be determined by the class the user is in.
						\end{itemize}		
					
					
				\item Drop connection
					\begin{itemize}
						\item Sessions will be checked periodically(every 10 minutes). Should a user be inactive for more than the stipulated period, the session will be terminated after the state of the system gets stored for possible retrieval.
					\end{itemize}		
							
				
				
			\end{itemize}			
			
		\end{itemize}
		\item Flexibility
		\begin{itemize}
			
			\item Tactics
			\begin{itemize}
				
				\item Contract based
					\begin{itemize}
						\item The use of the MVC architecture allows the system to be fairly flexible. The model will be the data(database) itself or actions on the database, the view will be the interface as experienced by the user and the controller will be the facilitator of the program flow logic. This modular format will allow changes to one aspect without greatly affecting the other aspects.
					\end{itemize}

			\end{itemize}			
			
		\end{itemize}
		
		\cleardoublepage
		
		\item Maintainability
		\begin{itemize}
			
			\item Tactics
			\begin{itemize}
				\item Localise changes
					\begin{itemize}
						\item Using MVC architecture, changes are localised. Service contracts will allow semantic coherence, the model component contains and encapsulates the functional core of the application, intermediaries in the form of the controller and the view, and runtime binding so that the view can be opened and closed dynamically.
					\end{itemize}
					
				\item Prevention of ripple effect 
				
					\begin{itemize}
						\item Intermediaries will be used to prevent ripple effects. The controller will realize the interface if the view and respond to accordingly, it will be the intermediary between the view and the model. The view will thus realize the model's interface and produce the output, making it an intermediary between the model and the controller.
					\end{itemize}

			\end{itemize}			
			
		\end{itemize}
		\item Auditability/monitorability
		\begin{itemize}
			
			\item Tactics
			\begin{itemize}
				\item Logging
					\begin{itemize}
						\item An audit trail will be maintained. This will be done by having a copy of each transaction applied to the data in the system along with the identifying information of the user.
					\end{itemize}					
			\end{itemize}			
			
		\end{itemize}
		\item Integrability
		\begin{itemize}

			
			\item Tactics
			\begin{itemize}
				
				\item Support communication channels
					\begin{itemize}
						\item The system will be made web browser independent to support the usual communication channels. Java will be used to integrate Android and the API.
					\end{itemize}
			\end{itemize}			
		
			
		\end{itemize}
		\item Cost
		\begin{itemize}
			
			\item Tactics
			\begin{itemize}
				\item Open source software
					\begin{itemize}
						\item The software used will be open software. Programming languages may include Java and python, frameworks like bootstrap and django and so on. 
					\end{itemize}
			\end{itemize}			
			
		\end{itemize}
		\item Usability
		\begin{itemize}
			
			\item Tactics
			\begin{itemize}

				\item User initiative
				
					\begin{itemize}
						\item The system will give the user feedback as to what the system is doing when the model updates the view/interface. The user will have the freedom to cancel or undo the operations that they are permitted to undertake.
					\end{itemize}				
				
				\item System initiative
					\begin{itemize}
						\item This is a design time tactic that is enforced by the MVC pattern. The view will be determined by the model which is designed by the architect to create a user friendly interface using HCI guidelines.
					\end{itemize}
			\end{itemize}			
			
		\end{itemize}
	\end{itemize}


\end{document}