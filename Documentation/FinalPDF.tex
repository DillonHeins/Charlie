\documentclass{article}

\usepackage[margin=2.5cm,left=2cm,includefoot]{geometry}
\usepackage{graphicx}
\usepackage{float}
\usepackage[space]{grffile}
\usepackage{hyperref}
\usepackage[export]{adjustbox}
\usepackage{multicol}

\usepackage{titlesec}

\setcounter{secnumdepth}{4}

\titleformat{\paragraph}
{\normalfont\normalsize\bfseries}{\theparagraph}{1em}{}
\titlespacing*{\paragraph}
{0pt}{3.25ex plus 1ex minus .2ex}{1.5ex plus .2ex}

% Header and footer
\usepackage{fancyhdr}
\pagestyle{fancy}

\rhead{COS301 - \LaTeX}
\lhead{Team Charlie}
\fancyfoot{}
\fancyfoot[R]{Page \thepage}

\renewcommand{\headrulewidth}{2pt}
\renewcommand{\footrulewidth}{1pt}
%

\begin{document}

	\begin{titlepage}
		\begin{center}
		
			\line(1,0){300}\\
			[6mm]
			\huge{
				\bfseries Functional Requirements and Software Architecture Specification
			}\\
			[2mm]
			\line(1,0){200}\\
			[15mm]
			\textsc{\large P.A.P.E.R.S (Publication And Papers Electronic Repository System)}\\
			[7.5mm]
			\textsc{\large University of Pretoria - Team Charlie}\\
			[20mm]
			\large{\textbf{Created By:}}\\
			[2mm]
			\large{
				\href{https://github.com/ClaudioMDS}{Claudio Da Silva - 14205892}\\
				\href{https://github.com/ArnoGrobler}{Arno Grobler - 14011396}\\
				\href{https://github.com/DillonHeins}{Dillon Heins - 14035538}\\
				\href{https://github.com/u13054903}{Charl Jansen Van Vuuren - 13054903}\\
				\href{https://github.com/pris264}{Priscilla Madigoe - 13049128}\\
				\href{https://github.com/BSchuld}{Bernhard Schuld - 10297902}\\
				\href{https://github.com/Keo11}{Keorapetse Shiko - 12231992}
			}\\
			[4cm]

		\href{https://github.com/DillonHeins/Charlie}{\textsc{\Large GitHub Repository - Team Charlie}\\[2mm]
		  For more information, please click here}
			
		\end{center}	
		\begin{flushright}
			\textsc{\large 17 February 2016}
		\end{flushright}
	\end{titlepage}
	
	\cleardoublepage
	\thispagestyle{empty}
	\tableofcontents
	\cleardoublepage
	\setcounter{page}{1}
	\section{Introduction}\label{sec:intro}
	P.A.P.E.R.S (Publication and Papers Electronic Repository System), a project proposed by the University of Pretoria's department of Computer Science, is a system dedicated to the management of research and publication records. The objective of the system is to create an environment where users of the system can maintain a list of publications belonging to them and where supervisors, such as a research leader or head of department, can oversee progress of said users. Users can belong to research groups and maintain a record of who contributed to a publication and thus the system is able to maintain and track the units earned by a user of the system for their publications. Only members of the department of Computer Science can be users of the system. 
	
	\subsection{General Uses}\label{subsec:generaluses}
		General uses of this system should include keeping track of:
            \begin{itemize}  
            
                \item Research papers and publications created by users of the system.
                \item Reporting.
                \item Research Groups.
                \item Running Costs.
                \item All historical publications that have been created.
                \item List of authors that helped contribute to the publications.
                \item List of users of the system.
                \item Units and respective conferences that the publications belong to.
            \end{itemize}
		\subsection{Purpose}\label{subsec:purpose}
			This document serves to explore the minutiae and requirements of the P.A.P.E.R system. This is including but not limited to the overall features of the system, interfaces, functionality, constraints and integration. The intended purpose of the system is for the developers to be used as a reference tool and for the clients as an informative resource. Any third party collaborators who have a need to use a reference to the P.A.P.E.R system may also use this documentation.
		\subsection{Structure}\label{subsec:structure}
			The structure of the document is to allow an overview of the system through domain models, use-case diagrams, pre and post conditions and a general analysis of the system as a whole.
		
		
	\cleardoublepage	
	\section{Vision}\label{sec:vision}
		The client has requested a system which allows researchers at the \textit{University of Pretoria}, specifically within the Computer Science Department, to keep track of the publications which they are currently actively involved with or working on.
		
		The system is required to keep track of historical publications so as to allow researches to maintain an archive of their work.
		
		The system should support the management of the multiple research groups within the department as well as allow the acting heads of the individual research groups to manage their group's members and publications.
		
		Ultimately this system is to be used by the acting Head of Department so as to be able to view all the research groups and their research output. It is a way for the department to ensure that the researchers are meeting their goals as well as the department's goals so as to ensure future funding for the department.\\
		[5mm]
		The typical usage scenarios for the desired output from this system would be:
		\begin{itemize}
			\item A UP staff member submitting a research paper to a conference, technical report or conference.
			\item The submission and acceptance of such a paper is what allows researchers to earn units.
			\item These units correspond with academic prestige as well as funding for the University of Pretoria and its researchers.
			\item Departments have predetermined goals which they set out to achieve each academic year.
			\item The ultimate desired output from this system is the ability to monitor the CS Department's researchers and their contribution towards earning these units.
			\item This allows the acting Head of Department to award researchers who achieve as well as take note of those who do not.
			% Desired output is also terminated papers - indicates who is not working
		\end{itemize}
	\cleardoublepage
	\section{Background}\label{sec:background}
		Reasons for the development of this project include but are not limited to:
		\begin{itemize}
			\item Research opportunities:
			\begin{itemize}
				\item Through the monitoring of units earned by staff members it enables the progression of research opportunities for the \textit{University of Pretoria's} Computer Science Department. By monitoring units earned the department is able to ensure that it meets its goals and is able to secure funding opportunities.
			\end{itemize}
			\item Opportunities to simplify some aspect of work:
			\begin{itemize}
				\item The method in use by the department currently is to have all researchers edit the same Microsoft Excel document as their manner of managing and tracking publications and units earned.
				\item This method is inefficient as well as error prone and has hence lead to the need to create this system as a means to replace it.
				\item This system aims to allow for all researchers to be able to manage their own publications in their own user space. It also allows for the Head of Department to no longer have to use an Excel document to create reports from, instead he/she would be able to use the system to do the work for him/her in a far more accurate and efficient manner.
			\end{itemize}
			\item Problems the client is currently facing:
			\begin{itemize}
				\item The problem of having all members of a department trying to collaborate on a single Excel document.
				\item The problem of having personal and academic information visible to all who have access to this document.
				\item The problem of managing this data in such a way as to get valuable and meaningful information out of it quickly and accurately.
			\end{itemize}
		\end{itemize}
		
	\cleardoublepage
	\section{Software Requirements}\label{sec:software}
	\subsection{Architecture Requirements}\label{subsec:architecture}
		\subsubsection{Architectural scope}\label{subsubsec:scope}
		
		\begin{figure}[H]
			\includegraphics[width=\linewidth]{../Diagrams/Architectural scope/Architectural scope.jpg}
			\caption{Architectural Scope - Use case for Architectural Scope (Further refined in use cases)}
		\end{figure}
		
		\par The system will be used to keep track of publications made for various research groups in the department, as well as keep track of the authors and the users in charge of things. From this we may later then generate reports on how various groups are doing as well as the department as a whole, and notify authors and users about upcoming deadlines and events.\\
		
		\par In the event of dealing with queries on database items such as users and publications, for example insertion and deletion, as well as viewing, the system must first validate that those items do or do not exist in said database. Once that is verified, the queries should be run and the result returned. On success the user should be notified, on failure an appropriate exception should be thrown. Persistence must be maintained throughout the database with very little fault tolerance.\\
		
		\par In dealing with reports, data being queried should be validated and the system should ensure that all queried data is recent and correct, halting any new additions during reporting. Reports should be properly archived for later retrieval and properly backed up regularly.\\
		
		\par In dealing with notifications, the system should ensure that all emails being used for every author does in fact exist and is a valid email. When sending this email, likely from the mail server at UP, the SMTP connection used must be ensured to be secured and the email should be sent through valid channels as to not appear as spam mail to other users. The system should also validate that it is using the latest deadline updates at all times.
		
		\subsubsection{Quality requirements}\label{subsubsec:quality}
		
		The core quality requirements for this system include in order:
		
		\begin{itemize}
			\item Cost
			\item Integrability,
			\item Security,
			\item Auditability,
			\item Reliability,
			\item Usability,
			\item Maintainability,
			\item Flexibility
			\item Performance, and
			\item Scalability. \newline
		\end{itemize}
		
		An in depth description of the various quality requirements: \newline
		
		\begin{itemize}
			\item Performance
			\begin{itemize}
				\item Performance in the system is of little worry as the system is not expected to grow to a large point in which it may slow down. The system should preferably be able to handle up to a hundred concurrent users at a time without major slow down effect and queries should be able to run within a matter of seconds.
			\end{itemize}
			\item Reliability
			\begin{itemize}
				\item The meta-data stored for each paper is required to be entirely correct at all times. To ensure this, we need to validate all possible input that can occur on the system in order to prevent any possible data inconsistencies. On the same note all data has to be persistent and remain in the database for the duration of the systems use, which means in all but the most extreme circumstances such as a database clean up for example, all entries must remain even if in a terminated state.
			\end{itemize}
			\item Scalability
			\begin{itemize}
				\item Scalability of this particular system is not of very particular importance. At the rate which the system will be used, chances of heavy fragmentation, data becoming too big or concurrency issues will be vary minimal.
			\end{itemize}
			\item Security
			\begin{itemize}
				\item Security is critical in the system as meta-data for unpublished papers can be sensitive information. Thus it is empirical that only authorised people have any access to the system at all. It is also vital to separate the concerns of each group from one another, allowing each group to only access the information from the groups they have access to. The Administrator should also be the only one allowed to make major changes to the system in anyway, and should be the only one authorised to make major persistence changes. 
			\end{itemize}
			\item Flexibility
			\begin{itemize}
				\item The system should be developed as to not be platform dependant or browser dependant in it's operation, allowing it to essentially be used from theoretically any device capable of an internet connection and HTML5, this includes iPhones which may still connect through safari as opposed to an iPhone application
				\item Flexibility may further be required in the event of change in units and venues for various existing conferences and journals, change in details for various users and authors registered on the system as well as flexibility for the addition or reduction of research groups.
			\end{itemize}
			\item Maintainability
			\begin{itemize}
				\item The system consists of a database which is required to stay consistent. To this end the system should contain well formatted documentation in order to properly maintain the system and ensure all data is in it's most consistent and reliable state.
			\end{itemize}
			\item Auditability
			\begin{itemize}
				\item All actions performed on the system should be audit-able via use of logging. A log entry should be created for each action such as adding or changing a publication for example. All actions performed in the system should be traceable to the user that performed them and possibly if so, where they performed it from in terms of IP as to have a form of accountability in the persistence of the meta-data.
			\end{itemize}
			\item Integrability
			\begin{itemize}
				\item The system should be easily integrable with the database system being used, allowing for simple queries in short amounts of time.
				\item The system should be integrable with a web front end that should allow users to easily and securely use all the features of the system that their authorisation level permits.
				\item The system should further be integrable with an Android front-end, which should allow for the exact same functionality of the web based front-end only designed ot be more mobile friendly.
			\end{itemize}
			\item Cost
			\begin{itemize}
				\item No proprietary software should be used in this system, and all licences for any software should be licensed under the open use policy.
			\end{itemize}
			\item Usability
			\begin{itemize}
				\item Users should easily be able to access the system securely from their most convenient location, either over their favourite web browser, the android application or via a mobile web page on their phone. All versions should allow full usage of all the features.
				\item Reports generated should be easily readable and allow for a logical and quick method of discerning various statistics from one another.
				\item Publications should be simple to view and easily distinguishable in type. One glance should allow a view of the most important details.
				\item Users should be easy to manage for the Administrator, allowing termination and reinstatement of users as well as addition of new ones to be a simple and streamlined process.
			\end{itemize}
		\end{itemize}
		
		\subsubsection{Integration and access channel requirements}\label{subsubsec:integration}
		
		\begin{itemize}
			\item The system must host a server to which API calls may be performed, to achieve this we use:
			\begin{itemize}
				\item Django (A Python based web framework allowing easy implementation of the MVC principal.)
				\item AngularJS (For easy unit testing functionality and dependency injection)
			\end{itemize}
			\item The system should make API calls via browser, the browser functionality and calling is done via:
			\begin{itemize}
				\item HTML5 (To ensure access to the latest and most stable versions of features)
				\item Javascript with a JQuery framework (For the functionality)
			\end{itemize}
			\item The system will require and android interface to interact with the API. To achieve this we make use of:
			\begin{itemize}
				\item JAVA (The standard language of every android application.)
				\item JQuery for mobile (A convenient method to manipulate data via JQuery on your android app.)
				\item Sencha touch (An HTML5 framework for Android.)
			\end{itemize}
			\item The system should be able to interact with the systems database, which will be hosted using:
			\begin{itemize}
				\item PostgreSQL (An open source relational database that integrates perfectly with the Django framework.)
			\end{itemize}
			\item The system should be able to send emails on a scheduled basis by sending mail from the CS mail server, for this we use:
			\begin{itemize}
				\item Django-sitemessage (A framework developed to allow easy scheduling of emails as well as the ability to retry an email on a failure.)\newline
			\end{itemize}
			
		\end{itemize}
		
		\begin{itemize}
			\item Protocols to be used include:
			\begin{itemize}
				\item HTTP
				\item SMTP
				\item TCP
			\end{itemize}
		\end{itemize}
		
		\subsubsection{Architectural constraints}\label{subsubsec:constraints}
		\begin{itemize}
			\item The system should be developed using only non-proprietary technologies.
			\item The system is to be developed for Linux based systems (Mostly Arch Linux) used by the Computer Science department, however preferably the system should run on any general use operating system.
			\item The system should be browser independent, thus the system should be accessed via any browser without issue, including but not limited to (Firefox, Chrome, Safari, Edge etc.)
			\item The system should be handled locally and should not rely on outside internet sources in order to function.
			\item The API created must be queried by both a web browser, as well as an Android application.
			\item The system created should allow for users to use the application via mobile, employing technologies such as bootstrap in order to keep it mobile friendly. This will always allow users of iOS and Windows mobile to access the system via channels such as Safari and Edge.
			\item The system is required to be concurrent and make use of concurrent methods specifically in it's functionality, allowing up to a minimum of a hundred concurrent users to be active at the same time
		\end{itemize}
		
		\subsection{Architectural Patterns or Styles}
		\subsection{Architectural tactics or strategies}
			\subsubsection{Introduction}
				Quality requirements have a significant influence on the software architecture of a system. Architectural tactics are techniques that an architect may use to comprehensively achieve the quality requirements. For each quality requirement listed, potential tactics will be mentioned.
			
			\subsubsection{Quality Requirements}
				\begin{itemize}
		\item Performance
		\begin{itemize}
			\item The system is fairly small and simple. Any form or action of enhancing performance may be relatively negligible but a couple of tactics can be implemented. 
			
			\item Tactics
			\begin{itemize}
					\item Spread load across resources
						
						\begin{itemize}
							\item Concurrency can be introduced. Processing different streams of events on different threads and creating additional threads to process different sets of activities can be done by processing requests in parallel. Appropriately allocating threads to resources can maximally exploit concurrency. 
							
						\end{itemize}
					
					\item Resource re-use, caching.
					
						\begin{itemize}
							\item Considering the Readers/Writers problem, multiple readers will have access to view the resource while only one writer can have access to edit the resource. Multiple copies of data can be maintained. Caching is a tactic that replicates data which would be on the same repository to reduce contention.
						\end{itemize}
						
			\end{itemize}		
		\end{itemize}
		
		\item Reliability
		\begin{itemize}
			
			\item Tactics
			\begin{itemize}
				\item Resource locking
				
					\begin{itemize}
						\item A resource or a part/section of the resource will be locked, the correct resource will be returned by making sure that the correct resource is requested to the database and only valid requests will be approved. After the resource is returned, only one user will be able to edit that resource. The other users will be notified about the lock by getting conflicts when they try to pull the resource.
					\end{itemize}
										
				\item Maintain backup
				
					\begin{itemize}
						\item The database will have a back up of all of the information that it stores.
					\end{itemize}
								

			\end{itemize}
						
		\end{itemize}
		\item Scalability
		\begin{itemize}
			
			\item Tactics
			\begin{itemize}

				\item Increasing capacity of communication channels
					
					\begin{itemize}
						\item Users will access the system through a website front-end on a web server maintained by the department. The site will be browser independent in order to increase communication channels. The Android system will also be used as a replacement for the website ans vice versa which allow use of mobile phones or tablets.
					\end{itemize}

			\end{itemize}
				
				
		\cleardoublepage
		
						
		\end{itemize}
		\item Security
		\begin{itemize}
			
			\item Tactics
			\begin{itemize}
				
				\item Authentication
					
					\begin{itemize}
						\item Users will be authenticated via their usernames and passwords by making use of server side validation.
					\end{itemize}
					
				\item Encryption
				
					\begin{itemize}
						 \item Passwords are to be hashed using at least sha256 and should be stored as such within the database along with a salt.
					\end{itemize}				 		
						 
				\item Authorisation
						\begin{itemize}
							\item Users will be grouped by user classes to ensure access control. Users rights to access and modification of data will be determined by the class the user is in.
						\end{itemize}		
					
					
				\item Drop connection
					\begin{itemize}
						\item Sessions will be checked periodically(every 10 minutes). Should a user be inactive for more than the stipulated period, the session will be terminated after the state of the system gets stored for possible retrieval.
					\end{itemize}
			\end{itemize}			
			
		\end{itemize}
		\item Flexibility
		\begin{itemize}
			
			\item Tactics
			\begin{itemize}
				
				\item Contract based
					\begin{itemize}
						\item The use of the MVC architecture allows the system to be fairly flexible. The model will be the data(database) itself or actions on the database, the view will be the interface as experienced by the user and the controller will be the facilitator of the program flow logic. This modular format will allow changes to one aspect without greatly affecting the other aspects.
					\end{itemize}

			\end{itemize}			
			
		\end{itemize}
		
		\item Maintainability
		\begin{itemize}
			
			\item Tactics
			\begin{itemize}
				\item Localise changes
					\begin{itemize}
						\item Using MVC architecture, changes are localised. Service contracts will allow semantic coherence, the model component contains and encapsulates the functional core of the application, intermediaries in the form of the controller and the view, and runtime binding so that the view can be opened and closed dynamically.
					\end{itemize}
					
				\item Prevention of ripple effect 
				
					\begin{itemize}
						\item Intermediaries will be used to prevent ripple effects. The controller will realize the interface if the view and respond to accordingly, it will be the intermediary between the view and the model. The view will thus realize the model's interface and produce the output, making it an intermediary between the model and the controller.
					\end{itemize}

			\end{itemize}			
			
		\end{itemize}
		\item Auditability/monitorability
		\begin{itemize}
			
			\item Tactics
			\begin{itemize}
				\item Logging
					\begin{itemize}
						\item An audit trail will be maintained. This will be done by having a copy of each transaction applied to the data in the system along with the identifying information of the user.
					\end{itemize}					
			\end{itemize}			
			
		\end{itemize}
		\item Integrability
		\begin{itemize}

			
			\item Tactics
			\begin{itemize}
				
				\item Support communication channels
					\begin{itemize}
						\item The system will be made web browser independent to support the usual communication channels. Java will be used to integrate Android and the API.
					\end{itemize}
			\end{itemize}			
		
			
		\end{itemize}
		\item Cost
		\begin{itemize}
			
			\item Tactics
			\begin{itemize}
				\item Open source software
					\begin{itemize}
						\item The software used will be open software. Programming languages may include Java and python, frameworks like bootstrap and django and so on. 
					\end{itemize}
			\end{itemize}			
			
		\end{itemize}
		\item Usability
		\begin{itemize}
			
			\item Tactics
			\begin{itemize}

				\item User initiative
				
					\begin{itemize}
						\item The system will give the user feedback as to what the system is doing when the model updates the view/interface. The user will have the freedom to cancel or undo the operations that they are permitted to undertake.
					\end{itemize}				
				
				\item System initiative
					\begin{itemize}
						\item This is a design time tactic that is enforced by the MVC pattern. The view will be determined by the model which is designed by the architect to create a user friendly interface using HCI guidelines.
					\end{itemize}
			\end{itemize}			
			
		\end{itemize}
	\end{itemize}
			
		
		\subsection{Reference Architecture and frameworks}\label{sec: ReferencArchitecture}
In this section we will discuss the reference architecture and frameworks that will be used in the P.A.P.E.R system.
	\subsubsection{An object-relational mapper}\label{subsec: ORM}
	Object-relational mapper is a technique of converting data between incompatible systems by means of object orientated programming. For example the conversion of database attributes into a database object. This allow the interacting object-orientated language to manipulate this database object with ease, in essence the data can be presented in a way that any object is presented in that programming language.
	
	Django uses Object-relational mapping(ORM) with regards to its Model-View-Controller (MVC) model.
	The data models and a relational database (Model) are manipulated with the ORM strategy to interact with the databases.
		
	
	\subsubsection{Advantages}
	\begin{itemize}
	\item
	The advantage of using object-relational mappers with databases in particular is that joins aren't used that often as object types can be followed by means of referencing pointers. 
	\item
	Relationships are also established by means of pointers which can increase efficiency for complex data.
	\item
	This approach works well for large amounts of data, as the object can be manipulated easily for each field.
	\end{itemize}
	
	\subsubsection{Disadvantages}
	
	\begin{itemize}
	\item
	 Inefficient when used with small databases as objects will still be created which might be less efficient than a quick lookup of those particular fields. 
	\end{itemize}
	
	\subsubsection{Application Server}\label{subsec: ApplicationServer}
	Software framework for web applications and a server to run the environment, is the principle that Application Server approach follows. An example of Application Server architecture framework is the Java EE framework. This architecture is based on the Layer model and Client-Server model and contains a service layer which is accessed by the developer.
	Django supports Java EE based Application Server.
	
	\subsubsection{Advantages}
	\begin{itemize}
	\item Scalability 
		\begin{itemize}
		\item Resources are allocated efficiently
		\item Objects reuse is ensured
		\end{itemize}
	\item Integrability
		\begin{itemize}
			\item Integrates well with REST
			\item Database integration is provided
		\end{itemize}
	\item Security
		\begin{itemize}
		\item Authentication and confidentiality is supported
		\end{itemize}
	\end{itemize}
	
	\subsubsection{Disadvantages}
	\begin{itemize}
	\item Uses a lot of system resources which might not necessarily be 
	\item Relies on server to be running at all times 
	\item Central point access can choke overall network's access to the server
	
	\end{itemize}

	\subsubsection{Django}\label{subsec: Django}
	Django is a web framework, written in Python which uses the Model-View-Controller architectural pattern, or MVC for short.
	Django's main aim is to provide a framework on which to build websites that are primarily based on complex databases.
	The use of Object-relational mapper in Django's MVC is described in \ref{subsec: ORM}. Django processes HTTP requests by means of a web templating system and regular-expression URL control (View and Controller).
	\\ \\
	\textbf{Django further includes:}
	\begin{itemize}
		\item Its own web server for developmental purposes.
		\item Form validation and storing of form data in database
		\item Caching framework with several cache methods
		\item Serialization system to produce and interpret XML and JSON representation of model instances.
		\item Python unit test framework
	\end{itemize}
	Django database support:
	\begin{itemize}
		\item PostgreSQL
		\item MySQL
		\item SQLite
		\item Oracle
		\item Microsoft SQL Server (through django-mssql)
	\end{itemize}
	Django can be used with:
	\begin{itemize}
		\item Python (Supported by default)
		\item JavaScript (through Swig)
		\item Ruby (through Liquid)
		\item Perl (through Template::Swig)
		\item PHP (Twig)
	\end{itemize}

	We prefer to use Django as it does have a slight learner curve, but will ease integrability with regards to our Android application.
	
	
	\subsubsection{Honorable Framework Mentions}
	\begin{itemize}
	
	
	\item\textbf{AngularJS}\label{subsubsec: AngularJS} \\
	Web-framework making use of client-side Model–View–Controller model. 
		AngularJS makes use of the MEAN stack for its front-end, consisting of MongoDB database, Express.js web application server framework, Angular.js itself, and Node.js runtime environment.
		We prefer to use a server side approach as this eases integrability and simplifies implementation.
	\item\textbf{Ruby on Rails}\label{subsubsec: Rails} \\
		Ruby based web-framework based on Ruby programming language. "Rails" uses a Model-View-Controller based model and emphasizes the use of JSON and XML for data transfer.
		Since Ruby has a high learning curve compared to Python for example, we prefer to use Django instead.
	\item\textbf{Zend Framework}\label{subsubsec: Zend}\\
	 Zend is an open source, object-oriented web application framework implemented in PHP 5. Zend supports multiple database systems and MVC is the preferred model of development.
	 Since Zend uses only Object-Orientated PHP5 we decided against this approach as the integration might be difficult for our Android Application.
	 \item\textbf{Bootstrap}\label{subsubsec: Bootstrap}\\
	 Bootstrap is a front-end framework for creating websites and applications. It is mainly used for interface development and design. Bootstrap can be used in HTML, CSS and JavaScript. 
	 We decided on a Server-side approach rather than a front-end approach such as Bootstrap.
	\end{itemize}
	
	\subsection{Access and Integration Channels}
		\subsubsection{Human Access Channels}
			Users will access the P.A.P.E.R.S system through a website front-end on a web server maintained by the Computer Science department. The website can be access off campus through any browser, i.e. Chrome, Mozilla Firefox, Internet Explorer, Safari and Edge. When on the website, the user will have to log in and depending on what user privilege the user has (lecturer, research leader, HOD) the user will access a different user interface, modelled to what the particular user is allowed to access. The website will be under strict standards-compliance that is having the website comply with the World Wide Web Consortium (W3C) (W3C, 2016) so that the website will run on every browser exactly the same. Users can also access the P.A.P.E.R.S system through an android application that uses the same web server as a back-end. The Android system can be used as a replacement for the website and vice versa. It should be able to work on mobile phones or tablets that supports the Android operating system.
		\subsubsection{Integration Channels}
			\paragraph{LDAP}
				LDAP (Lightweight Directory Access Protocol), is an Internet protocol that email and other programs use to look up information on a server (Gracion Software, 2016). In the case of the P.A.P.E.R.S system LDAP can be used in particular to storing and retrieving user information stored in the database, such as the user’s username and password. The LDAP protocol runs a layer above the TCP/IP layer and provides the mechanisms to connect to, search, and modify Internet directories, such is found in the P.A.P.E.R.S system (Microsoft, 2016). LDAP supports C and C++ programming languages. The aforementioned database will be the main backbone of the system and thus will include quality requirements:
			\begin{itemize}  
        		\item Performance: Accessing data from the database needs to happen as fast as possible. This means an equally fast protocol needs to be used to be used. Since the website and Android application can be run and used offsite from the University of Pretoria, SCP will be used to quickly move around data. FTP could have been used, but it is not as secure as SCP which uses the SSH protocol for authentication (LiquidWeb, 2016). SFTP (a more secure version of FTP) could also be used but because speed as an issue, SCP will be used instead as it is faster (LiquidWeb, 2016).
        		\item Reliability: The system should not have downtime and thus should have a secure and reliable connection to the LDAP system.
       			\item Scalability: The database must be able to store and manage a large amount of entries that include relationships to all entities such as users, publications, authors, etc. 
        		\item Auditability: Any change to the system through LDAP and SCP should be logged in a log file. This includes what queries were made, who made them, what time and a trace of the actions performed.
        		\item Flexibility: This should be shown in the database as it being in a state where any change done to the database should not in any way affect the system as a whole. This requires a high level of normalisation in the database to prevent any data anomalies that could occur from using the system.
        		\item Affordability: Any request made by the LDAP system, made by the user, should not affect the performance of the system as a whole and should be affordable in a sense to not prevent any other user concurrently using the system from not being able to perform their actions optimally.
    		\end{itemize}
    		
    		\paragraph{HTTPS}
    			HTTP will be used for the web services and in particular HTTPS will be used to make the system more secure. The reason being is that traffic sent over HTTP is highly confidential (user’s usernames and passwords) and thus need to be protected. As the whole web based approach will require the HTTPS protocol, it would need quality requirements:
     \begin{itemize}
        \item Performance: the actual difference in speed compared to a normal HTTP is in insignificant, in fact even a little bit   faster (see annex A) but can take longer in regards to the SSL handshake request that has to be made. This is still something that can be overlooked as security has a higher priority than speed.
        \item Reliability: Since HTTPS is not reliable enough in itself as a connection is never maintained. Thus TCP in conjunction with HTTPS can be used. This will allow you to wait for responses that could inform you if the server has received all the information. 
        \item Scalability: HTTPS must be able to handle many requests sent and received and not slow down the system as a whole when doing so.
        \item Auditability: Every HTTPS request needs to  logged in an appropriate log file to see what has been requested
        \end{itemize}
        	
        	\paragraph{SMTP}
				SMTP (Simple Mail Transfer Protocol) will be used to send email notifications to users of the system. This falls under the notification domain of the system and will be linked to the deadlines set by users in the database. SMPT uses Port 25 to send emails but Port 465 can be used instead if encryption needs to be used to send emails securely (SiteGround, 2016). 
				
			\paragraph{Other Integration Channels}
				\begin{itemize}
           			\item Javascript/Jquery could be used on the front end to provide a more user friendly and interactive system and to prevent any unauthorised changes to the system. Bootstrap can work on the front end to help maintain a common and consistent user interface.
			         \item JPA (Java Persistence API) is Java specification for accessing, persisting, and managing data in a Java based database that  is what is chosen to work on the back-end. 
			          \item SQL (Structured Query language) can be used if a relational database is used in the place of a java based database. noSQL can thus be used if the database is non relational
				\end{itemize}
	\subsection{Technologies}

	For each of the following sections, all technologies we considered are listed.
	
	\subsubsection{Programming Languages}
		\begin{itemize}
			\item\textbf{Java}\\
			Java is a general purpose, high-level programming language developed by Sun Microsystems. It is concurrent, class based and object-oriented. It was specifically designed to have as few dependencies as possible.
				
				\begin{itemize}			
					\item Advantages:
						\begin{itemize}
							\item Easy to use
							\item Syntax is derived from C and C++
							\item Comprehensive documentation
						\end{itemize}
						
					\item Disadvantages:
						\begin{itemize}
							\item Memory Inefficient
						\end{itemize}
				\end{itemize}
				
			\item\textbf{JavaScript} \\
			
			JavaScript is a high-level programming language that is, alongside HTML and CSS, one of the three essential technologies that allow content production for the World Wide Web.
				
				\begin{itemize}
					\item Advantages:
						\begin{itemize}
							\item  Because JavaScript is client-side, there is no delay by having to wait for a server response
							\item Easy to learn and implement
						\end{itemize}
						
					\item Disadvantages:
						\begin{itemize}
							\item Security, the code being executed on the client-side is susceptible to malicious exploitation
						\end{itemize}
				\end{itemize}
				
			\item\textbf{Python} \\
			Python is a high-level programming language that emphasizes code readability.
			
				\begin{itemize}
					\item Advantages:
						\begin{itemize}
							\item Efficiency, Python allows a programmer to solve the same problem in fewer lines of code than in other languages such as Java
							\item Easy to read
						\end{itemize}
						
					\item Disadvantages:
						\begin{itemize}
							\item Syntax differs from conventional languages such as Java or C++, such as the omission of the semicolon
							\item Indentation dictates blocks of code, a single wrong indentation will produce undesired or unexpected results from your code
						\end{itemize}
				\end{itemize}
				
			\item\textbf{PHP} \\
			PHP (Hypertext Preprocessor) is a server-side scripting language designed for web development.
				
				\begin{itemize}
					\item Advantages:
						\begin{itemize}
							\item Works well with databases
							\item Popular, most problems encountered have already been solved by other developers
						\end{itemize}
						
					%\item Disadvantages:
					%	\begin{itemize}
					%		\item ?
					%	\end{itemize}
				\end{itemize}
				
			\item\textbf{C} \\
			
				\begin{itemize}
					\item Advantages:
						\begin{itemize}
							\item Fast run-time performance
						\end{itemize}
						
					\item Disadvantages:
						\begin{itemize}
							\item Low-level language, not ideal for applications or web development
						\end{itemize}
				\end{itemize}
				
			\item\textbf{C++} \\
			
				\begin{itemize}
					\item Advantages:
						\begin{itemize}
							\item Powerful language
							\item 
						\end{itemize}
						
					\item Disadvantages:
						\begin{itemize}
							\item No garbage collection, memory management has to be implemented by the programmer
							\item Complex language
						\end{itemize}
				\end{itemize}
				
			\item\textbf{HTML} \\ %strictly speaking it is a markup language, but I think it fits with this section.
			HTML (HyperText Markup Language) is the standard markup language to create web pages. It dictates the content of a web page. Alongside JavaScript and CSS, it is one of the three essential technologies that allow content production for the World Wide Web.
				\begin{itemize}
					\item Advantages:
						\begin{itemize}
							\item Standardized, it is the standard markup language to create web pages
							\item Easy to learn
						\end{itemize}
					\item Disadvantages:
						\begin{itemize}
							\item Different web browsers may render the page differently
							\item Bland, it has limited styling capability.
						\end{itemize}
				\end{itemize}

		\end{itemize}
	\subsubsection{Frameworks}
	\begin{itemize}

		\item\textbf{Ajax} \\
		AJAX (Asynchronous Javascript and XML) is a group of technologies used to create asynchronous web applications. It is used to change the content of a web page dynamically without having to reload the entire web page
			
		\item\textbf{AngularJS} \\
		Described in section \ref{subsubsec: AngularJS}
			
		\item\textbf{Bootstrap} \\
		Described in section \ref{subsubsec: Bootstrap}
			
		\item\textbf{Django} \\
		Described in detail in section \ref{subsec: Django}
	\end{itemize}
	
	\subsubsection{Libraries}\
	\begin{itemize}
			\item\textbf{jQuery} \\
			jQuery is a cross-platform JavaScript library designed to simplify client-side scripting. It is the most popular JavaScript library in use today
	\end{itemize}
	\subsubsection{Protocols}
		All protocols listed below were discussed in section \ref{sec: Integration channels}
			\begin{itemize}
				\item LDAP (Lightweight Directory Access Protocol)
				\item HTTPS
				\item HTTP
				\item SMTP	
			\end{itemize}
	
	\subsubsection{Database Systems}
		The System will utilize a relational database and will use PostgreSQL to handle queries.
		
	\subsubsection{Operating Systems}
		The System will be created to primarily work on Linux as the entire Computer Department is running on Linux. The System will however be compatible with other operating systems such as Windows, Apple OS.
		
		
	\cleardoublepage
	\section{Functional Requirements and Application Design}\label{sec:functional}
		\subsection{Use Case Prioritization}
			\subsubsection{Publication Sub-System}\label{subsubsec:priority-paper}
				Handles the adding, editing and terminating of papers.\\
				[3mm]
				\textbf{User:}
				\begin{itemize}
					\item \textit{Critical:}
					\begin{itemize}
						\item add/edit own paper entry
						\item add/remove authors from a paper
						\item create new author
						\item terminate/resume own paper
						\item add/remove conference/journal to paper
						\item create new conference/journal
					\end{itemize}
					
					\item \textit{Important:}
					\begin{itemize}
						\item edit own paper progress
					\end{itemize}
				\end{itemize}
				\textbf{Head of Research:}
				\begin{itemize}
					\item \textit{Critical:}
					\begin{itemize}
						\item search all papers within own research group
					\end{itemize}
				\end{itemize}
				\textbf{Admin:}
				\begin{itemize}
					\item \textit{Critical:}
					\begin{itemize}
						\item search all papers on system
					\end{itemize}
					
					\item \textit{Important:}
					\begin{itemize}
						\item purge paper from the system
					\end{itemize}
				\end{itemize}
			\subsubsection{User Sub-System}\label{subsubsec:priority-user}
				Handles all actions associated with user accounts such as login, viewing and editing profiles, as well as privileged actions such as adding/removing users to the system and adding/removing users from a research group.\\
				[3mm]
				\textbf{User:}
				\begin{itemize}
					\item \textit{Critical:}
					\begin{itemize}
						\item login
						\item logout
						\item view own user profile
						\item edit own user profile
					\end{itemize}
				\end{itemize}
				\textbf{Head of Research:}
				\begin{itemize}
					\item \textit{Critical:}
					\begin{itemize}
						\item add/remove users from own research group
						\item search/view all users in own research group
					\end{itemize}
				\end{itemize}
				\textbf{Admin:}
				\begin{itemize}
					\item \textit{Critical:}
					\begin{itemize}
						\item add new user to system
						\item search/view all users in any research group
						\item add/remove users from any research group
					\end{itemize}
					
					\item \textit{Important:}
					\begin{itemize}
						\item purge user from system
						\item set user as active/inactive
					\end{itemize}
				\end{itemize}
			\subsubsection{Notification Sub-System}\label{subsubsec:priority-notification}
				Handles the adding and editing of notifications within the system, as well the process of creating and sending notifications to relevant users of the system.\\
				[3mm]
				\textbf{User:}
				\begin{itemize}					
					\item \textit{Important:}
					\begin{itemize}
						\item update paper deadline
					\end{itemize}
				\end{itemize}
				\textbf{System:}
				\begin{itemize}
					\item \textit{Important:}
					\begin{itemize}
						\item send deadline update notification
						\item send deadline notification
					\end{itemize}
				\end{itemize}
			\subsubsection{Reporting Sub-System}\label{subsubsec:priority-report}
				Handles the generation of reports with regards to all relevant information contained by the system.\\
				[3mm]
				\textbf{Head of Research:}
				\begin{itemize}					
					\item \textit{Important:}
					\begin{itemize}
						\item generate report for research group
					\end{itemize}
				\end{itemize}
				\textbf{Admin:}
				\begin{itemize}
					\item \textit{Important:}
					\begin{itemize}
						\item generate report for department
					\end{itemize}
					\item \textit{Nice-to-Have:}
					\begin{itemize}
						\item dump database to file
					\end{itemize}
				\end{itemize}
			\subsubsection{Group-Control Sub-System}\label{subsubsec:priority-group}
				Handles the creating and removing of research groups, as well as the allocating of heads of research groups and the editing of individual research groups associated metadata.\\
				[3mm]
				\textbf{Head of Research:}
				\begin{itemize}					
					\item \textit{Critical:}
					\begin{itemize}
						\item edit research group information
					\end{itemize}
				\end{itemize}
				\textbf{Admin:}
				\begin{itemize}
					\item \textit{Critical:}
					\begin{itemize}
						\item add research group
						\item edit any research group's information
						\item allocate/change head of research group
					\end{itemize}
					\item \textit{Important:}
					\begin{itemize}
						\item set research group as active/inactive
					\end{itemize}
				\end{itemize}
				
			\subsubsection{Logging Sub-System}\label{subsubsec:priority-logging}
				Handles the logging of all important interactions which take place on the system.\\
				[3mm]
				\textbf{Admin:}
				\begin{itemize}
					\item \textit{Critical:}
					\begin{itemize}
						\item download and view log text files
					\end{itemize}
				\end{itemize}
				\textbf{Date and Time:}
				\begin{itemize}
					\item \textit{Important:}
					\begin{itemize}
						\item move current log file to backup
					\end{itemize}
				\end{itemize}
			
		\cleardoublepage
		\subsection{Service Contracts}\label{subsec:servicecontracts}
			\subsubsection{Publication Sub-System}\label{subsubsec:publication}
				Handles the adding, editing and terminating of papers.\\
				[3mm]
				\textbf{User:}
				\begin{itemize}
					\item \textbf{Add/Edit Own Paper Entry} \hfill \textit{Priority: Critical}
					\begin{itemize}
						\item A user of the system should be able to add or edit a paper entry associated with their user account.
						\item \textbf{Pre-conditions:}
						\begin{itemize}
							\item If adding:
							\begin{itemize}
								\item A user should have a registered account and be logged in.
							\end{itemize}
							\item If editing:
							\begin{itemize}
								\item A user must be logged in.
								\item The logged in user must be an author or co-author of the paper.
								\item The paper must not have been terminated.
							\end{itemize}
						\end{itemize}
						\item \textbf{Post-conditions:}
						\begin{itemize}
							\item If adding:
							\begin{itemize}
								\item A paper entry and all its details should be entered into the database of the system and be associated with a particular registered user.
							\end{itemize}
							\item If editing:
							\begin{itemize}
								\item The metadata in the database associated with the paper should be updated.
							\end{itemize}													
							\item The act of creating/editing a new paper entry should be logged.
						\end{itemize}
						\item \textbf{Request and Results Data Structures:}
						\begin{figure}[H]
							\includegraphics[width=\linewidth]{../Diagrams/ServiceContracts/Publication subsystem/AddPublication.jpg}
							\caption{Service Contract - Add Own Paper Entry}
						\end{figure}
						\begin{figure}[H]
							\includegraphics[width=\linewidth]{../Diagrams/ServiceContracts/Publication subsystem/EditPublication.jpg}
							\caption{Service Contract - Edit Own Paper Entry}
						\end{figure}
					\end{itemize}
					
					\item \textbf{Add/Remove Authors from a Paper:} \hfill \textit{Priority: Critical}
					\begin{itemize}
						\item A user should be able to add and remove authors from a paper which they have created or are a co-author of.
						\item \textbf{Pre-conditions:}
						\begin{itemize}
							\item The user must be logged in.
							\item The user must be the author or a co-author of the paper.
						\end{itemize}
						\item \textbf{Post-conditions:}
						\begin{itemize}
							\item There should be an ordered list of unlimited size associated with the paper consisting of the details of the author and co-authors stored in the database.
							\item The action should be logged.
						\end{itemize}
						\item \textbf{Request and Results Data Structures:}
						\begin{figure}[H]
							\includegraphics[width=\linewidth]{../Diagrams/ServiceContracts/Publication subsystem/AddRemoveAuthorFromPaper.jpg}
							\caption{Service Contract - Add/Remove Author from Paper}
						\end{figure}
					\end{itemize}
					
					\item \textbf{Add/Edit Author:} \hfill \textit{Priority: Critical}
					\begin{itemize}
						\item A user should have the option to add the details of a new author into the system if the author does not already exist within the system as well as edit existing authors.
						\item \textbf{Pre-conditions:}
						\begin{itemize}
							\item The user must be logged in.
							\item The author must not already exist in the case of adding.
							\item If the author does exist they should be edited.
						\end{itemize}
						\item \textbf{Post-conditions:}
						\begin{itemize}
							\item The author and his/her associated details have been added to the system's database in the case of adding.
							\item The author and his/her associated details should have been updated and stored in the database in the case of editing.
							\item The actions should be logged.
						\end{itemize}
						\item \textbf{Request and Results Data Structures:}
						\begin{figure}[H]
							\includegraphics[width=\linewidth]{../Diagrams/ServiceContracts/Publication subsystem/AddAuthor.jpg}
							\caption{Service Contract - Add Author to System}
						\end{figure}
						\begin{figure}[H]
							\includegraphics[width=\linewidth]{../Diagrams/ServiceContracts/Publication subsystem/EditAuthor.jpg}
							\caption{Service Contract - Edit Author}
						\end{figure}
					\end{itemize}
					
					\cleardoublepage
					\item \textbf{Edit Own Paper Progress:} \hfill \textit{Priority: Important}
					\begin{itemize}
						\item A user should have the ability to indicate using a percentage how far they are currently in terms of progress on their paper.
						\item \textbf{Pre-conditions:}
						\begin{itemize}
							\item A user should be logged in.
							\item The logged in user must be an author or co-author of the paper.
							\item The paper must not have been terminated.
						\end{itemize}
						\item \textbf{Post-conditions:}
						\begin{itemize}
							\item The field indicating progress within the database must have been updated.
							\item The change in progress should be depicted visually to the user.
							\item The action should be logged.
						\end{itemize}
						\item \textbf{Request and Results Data Structures:}
						\begin{figure}[H]
							\includegraphics[width=\linewidth]{../Diagrams/ServiceContracts/Publication subsystem/EditPaperProgress.jpg}
							\caption{Service Contract - Edit Paper Progress}
						\end{figure}
					\end{itemize}			
					
					\cleardoublepage
					\item \textbf{Terminate/Resume Own Paper:} \hfill \textit{Priority: Critical}
					\begin{itemize}
						\item A user must be able to terminate (make inactive) a paper which they feel they are not going to be completing in the near future. They must also be able to resume this paper at a later stage when they want to continue work on it.
						\item \textbf{Pre-conditions:}
						\begin{itemize}
							\item A user must be logged in.
							\item The logged in user must be an author or co-author of the paper.
						\end{itemize}
						\item \textbf{Post-conditions:}
						\begin{itemize}
							\item If terminating:
							\begin{itemize}
								\item The paper has been marked as terminated within the database.
								\item The action of terminating the paper must have been logged.
								\item The paper must no longer be able to be edited by the user.
							\end{itemize}
							\item If resuming:
							\begin{itemize}
								\item The paper has been unmarked as terminated within the database.
								\item The action of resuming the paper must have been logged.
								\item The paper must be available to edit.
							\end{itemize}
						\end{itemize}
						\item \textbf{Request and Results Data Structures:}
						\begin{figure}[H]
							\includegraphics[width=\linewidth]{../Diagrams/ServiceContracts/Publication subsystem/TerminatePaper.jpg}
							\caption{Service Contract - Terminate Paper}
						\end{figure}
						\begin{figure}[H]
							\includegraphics[width=\linewidth]{../Diagrams/ServiceContracts/Publication subsystem/ResumePaper.jpg}
							\caption{Service Contract - Resume Paper}
						\end{figure}
					\end{itemize}
					
					\cleardoublepage
					\item \textbf{Add/Remove Conference/Journal to Paper:} \hfill \textit{Priority: Critical}
					\begin{itemize}
						\item Whilst creating or editing a paper a user should be able to associate a particular conference or journal with their paper as well as be able to remove an already associated conference or journal whilst editing a paper.
						\item \textbf{Pre-conditions:}
						\begin{itemize}
							\item The user must be logged in.
							\item The user must be an author or co-author of the paper in question.
						\end{itemize}
						\item \textbf{Post-conditions:}
						\begin{itemize}
							\item The paper should be associated with a particular conference or journal or the paper should have had its association removed.
							\item The action should be logged.
						\end{itemize}
						\item \textbf{Request and Results Data Structures:}
						\begin{figure}[H]
							\includegraphics[width=\linewidth]{../Diagrams/ServiceContracts/Publication subsystem/AddRemovePublicationTypeFromPaper.jpg}
							\caption{Service Contract - Add/Remove Publication Type from Paper}
						\end{figure}
					\end{itemize}									
					
					\cleardoublepage
					\item \textbf{Add/Edit Conference/Journal:} \hfill \textit{Priority: Critical}
					\begin{itemize}
						\item A user must be able to fill in the details of a particular conference or journal if it does not already exist within the system. If it does exist the user should be able to edit it.
						\item \textbf{Pre-conditions:}
						\begin{itemize}
							\item The user must be logged in.
							\item The conference or journal must not already exist.
						\end{itemize}
						\item \textbf{Post-conditions:}
						\begin{itemize}
							\item The conference or journal and all its details have been added to the system's database in the case of adding.
							\item In the case of editing the conference/journal must have been updated within the database.
							\item The actions have been logged.
						\end{itemize}
						\item \textbf{Request and Results Data Structures:}
						\begin{figure}[H]
							\includegraphics[width=\linewidth]{../Diagrams/ServiceContracts/Publication subsystem/AddPublicationType.jpg}
							\caption{Service Contract - Add Publication Type}
						\end{figure}
						\begin{figure}[H]
							\includegraphics[width=\linewidth]{../Diagrams/ServiceContracts/Publication subsystem/EditPublicationType.jpg}
							\caption{Service Contract - Edit Publication Type}
						\end{figure}
					\end{itemize}								
				\end{itemize}
				\cleardoublepage
				\textbf{Head of Research:}
				
				\begin{itemize}
					\item \textbf{Search All Papers Within Own Research Group} \hfill \textit{Priority: Critical}
					\begin{itemize}
						\item The head of a research group should have access to all the papers associated with their particular research group.
						\item \textbf{Pre-conditions:}
						\begin{itemize}
							\item The user must be logged in.
							\item The user must be the head of the particular research group they would like to observe.
						\end{itemize}
						\item \textbf{Post-conditions:}
						\begin{itemize}
							\item The user should be able to view all metadata associated with the papers of their research group.
							\item The action should be logged.
						\end{itemize}
						\item \textbf{Request and Results Data Structures:}
						\begin{figure}[H]
							\includegraphics[width=\linewidth]{../Diagrams/ServiceContracts/Publication subsystem/SearchAllPapersInGroup.jpg}
							\caption{Service Contract - Search All Papers in Group}
						\end{figure}
					\end{itemize}
				\end{itemize}
				
				\cleardoublepage
				\textbf{Admin:}
				
				\begin{itemize}
					\item \textbf{Search All Papers on System} \hfill \textit{Priority: Critical}
					\begin{itemize}
						\item An admin user should be able to view all papers and their metadata on the system.
						\item \textbf{Pre-conditions:}
						\begin{itemize}
							\item The user must be logged in.
							\item The user must have the privilege rights assigned to an admin.
						\end{itemize}
						\item \textbf{Post-conditions:}
						\begin{itemize}
							\item The user should be able to view all metadata associated with all the papers stored on the system.
							\item The action should be logged.
						\end{itemize}
						\item \textbf{Request and Results Data Structures:}
						\begin{figure}[H]
							\includegraphics[width=\linewidth]{../Diagrams/ServiceContracts/Publication subsystem/SearchAllPapers.jpg}
							\caption{Service Contract - Search All Papers}
						\end{figure}
					\end{itemize}
					
					\cleardoublepage
					\item \textbf{Purge Paper From the System} \hfill \textit{Priority: Important}
					\begin{itemize}
						\item An admin user should be able to completely remove a paper from the system if they find the need to do so.
						\item \textbf{Pre-conditions:}
						\begin{itemize}
							\item The user must be logged in.
							\item The user must have the privilege rights assigned to an admin.
						\end{itemize}
						\item \textbf{Post-conditions:}
						\begin{itemize}
							\item The paper and all of its metadata should be permanently removed from the system.
							\item The action should be logged.
						\end{itemize}
						\item \textbf{Request and Results Data Structures:}
						\begin{figure}[H]
							\includegraphics[width=\linewidth]{../Diagrams/ServiceContracts/Publication subsystem/PurgePaper.jpg}
							\caption{Service Contract - Purge Paper}
						\end{figure}
					\end{itemize}
				\end{itemize}
			\cleardoublepage
			\subsubsection{User Sub-System}\label{subsubsec:user}
				Handles all actions associated with user accounts such as login, viewing and editing profiles, as well as privileged actions such as adding/removing users to the system and adding/removing users from a research group.\\
				[3mm]
				\textbf{User:}
				\begin{itemize}
					\item \textbf{Login} \hfill \textit{Priority: Critical}
					\begin{itemize}
						\item A registered user of the system must be able to login to the system using their private credentials.
						\item \textbf{Pre-conditions:}
						\begin{itemize}
							\item The user must not be logged in.
							\item The identifying field used (email address) must match one of an active, registered user of the system.
							\item The password entered must match the password associated with the identifying field within the system.
						\end{itemize}
						\item \textbf{Post-conditions:}
						\begin{itemize}
							\item The system must begin a session for the user.
							\item The user must gain access to the system according to the set of privileges associated with their account.		
							\item The action should be logged.
						\end{itemize}
						\item \textbf{Request and Results Data Structures:}
						\begin{figure}[H]
							\includegraphics[width=\linewidth]{../Diagrams/ServiceContracts/User subsystem/Login.jpg}
							\caption{Service Contract - Login}
						\end{figure}
					\end{itemize}
					
					\cleardoublepage
					\item \textbf{Logout} \hfill \textit{Priority: Critical}
					\begin{itemize}
						\item A user must be able to end their session and log out of the system.
						\item \textbf{Pre-conditions:}
						\begin{itemize}
							\item A user must be logged in.
						\end{itemize}
						\item \textbf{Post-conditions:}
						\begin{itemize}
							\item The user's session must be ended and the user must no longer be logged in.
						\end{itemize}
						\item \textbf{Request and Results Data Structures:}
						\begin{figure}[H]
							\includegraphics[width=\linewidth]{../Diagrams/ServiceContracts/User subsystem/Logout.jpg}
							\caption{Service Contract - Logout}
						\end{figure}
					\end{itemize}
					
					\cleardoublepage
					\item \textbf{View Own User Profile} \hfill \textit{Priority: Critical}
					\begin{itemize}
						\item A logged in user should be able to see their own profile and all details associated with it. The user should also be able to view all papers associated with their account.
						\item \textbf{Pre-conditions:}
						\begin{itemize}
							\item The user must be logged in.
							\item The user's profile being viewed must be the same as the logged in user.
						\end{itemize}
						\item \textbf{Post-conditions:}
						\begin{itemize}
							\item The information associated with the user's account should be displayed to them.
							\item The action should be logged.
						\end{itemize}
						\item \textbf{Request and Results Data Structures:}
						\begin{figure}[H]
							\includegraphics[width=\linewidth]{../Diagrams/ServiceContracts/User subsystem/ViewOwnUserProfile.jpg}
							\caption{Service Contract - View Own User Profile}
						\end{figure}
					\end{itemize}
					
					\cleardoublepage
					\item \textbf{Edit Own User Profile} \hfill \textit{Priority: Critical}
					\begin{itemize}
						\item A logged in user should be able to edit all details associated with their user account.
						\item \textbf{Pre-conditions:}
						\begin{itemize}
							\item The user must be logged in.
							\item The user profile being edited must belong to the user editing it.
						\end{itemize}
						\item \textbf{Post-conditions:}
						\begin{itemize}
							\item The updated information associated with the user's account should be stored in the system.
							\item The action should be logged.
						\end{itemize}
						\item \textbf{Request and Results Data Structures:}
						\begin{figure}[H]
							\includegraphics[width=\linewidth]{../Diagrams/ServiceContracts/User subsystem/EditOwnUserProfile.jpg}
							\caption{Service Contract - Edit Own User Profile}
						\end{figure}
					\end{itemize}							
				\end{itemize}
				\cleardoublepage
				\textbf{Head of Research:}
				
				\begin{itemize}
					\item \textbf{Add/Remove Users from Own Research Group} \hfill \textit{Priority: Critical}
					\begin{itemize}
						\item The head of a research group should be able to manage the users associated with that group.
						\item \textbf{Pre-conditions:}
						\begin{itemize}
							\item The user must be logged in.
							\item The user must have the privileges associated with a Head of Research.
							\item The user must be the Head of Research of the particular group which they are wanting to manage.
						\end{itemize}
						\item \textbf{Post-conditions:}
						\begin{itemize}
							\item The user must have been able to add other registered users into their research group.
							\item The user must have been able to remove other registered users from their research group.
							\item The action should be logged.
						\end{itemize}
						\item \textbf{Request and Results Data Structures:}
						\begin{figure}[H]
							\includegraphics[width=\linewidth]{../Diagrams/ServiceContracts/User subsystem/AddUserToOwnGroup.jpg}
							\caption{Service Contract - Add User to Own Group}
						\end{figure}
						\begin{figure}[H]
							\includegraphics[width=\linewidth]{../Diagrams/ServiceContracts/User subsystem/RemoveUserFromOwnGroup.jpg}
							\caption{Service Contract - Remove User from Own Group}
						\end{figure}
					\end{itemize}
					
					\cleardoublepage
					\item \textbf{Search/View All Users in Own Research Group} \hfill \textit{Priority: Critical}
					\begin{itemize}
						\item The head of a research group should be able to search all of the users within their research group. They should also be able to access the profiles and information associated with the users within their research group. In particular the head of a research group should be able to view all paper entries associated with each user.
						\item \textbf{Pre-conditions:}
						\begin{itemize}
							\item The user must be logged in.
							\item The user must be the Head of Research of the particular group which they are wanting to search/view.
						\end{itemize}
						\item \textbf{Post-conditions:}
						\begin{itemize}
							\item The user must have been able to view all of the users associated with their research group.
							\item The head of the research group can access and view all profile information (including paper entries) for each of the users associated with their research group.
							\item The action should be logged.
						\end{itemize}
						\item \textbf{Request and Results Data Structures:}
						\begin{figure}[H]
							\includegraphics[width=\linewidth]{../Diagrams/ServiceContracts/User subsystem/SearchUsersInOwnGroup.jpg}
							\caption{Service Contract - Search Users in Own Group}
						\end{figure}
						\begin{figure}[H]
							\includegraphics[width=\linewidth]{../Diagrams/ServiceContracts/User subsystem/ViewUsersInOwnGroup.jpg}
							\caption{Service Contract - View User in Own Group}
						\end{figure}
					\end{itemize}
				\end{itemize}
				
				\cleardoublepage
				\textbf{Admin:}
				
				\begin{itemize}
					\item \textbf{Add New User to System} \hfill \textit{Priority: Critical}
					\begin{itemize}
						\item Admin users should have the ability to add new users into the system.
						\item \textbf{Pre-conditions:}
						\begin{itemize}
							\item The user must be logged in.
							\item The user must have the privileges associated with an admin account.
						\end{itemize}
						\item \textbf{Post-conditions:}
						\begin{itemize}
							\item A new user with all their metadata as well as privileges should be added into the system.
							\item The action should be logged.
						\end{itemize}
						\item \textbf{Request and Results Data Structures:}
						\begin{figure}[H]
							\includegraphics[width=\linewidth]{../Diagrams/ServiceContracts/User subsystem/AddNewUser.jpg}
							\caption{Service Contract - Add New User}
						\end{figure}
					\end{itemize}
					
					\cleardoublepage
					\item \textbf{Search/View All Users in Any Research Group} \hfill \textit{Priority: Critical}
					\begin{itemize}
						\item Admin users should be able to search through all of the users within the system as well as view their profiles and all associated data and papers.
						\item \textbf{Pre-conditions:}
						\begin{itemize}
							\item The user must be logged in.
							\item The user must have the privileges associated with an admin account.
						\end{itemize}
						\item \textbf{Post-conditions:}
						\begin{itemize}
							\item The admin user should have been able to search and locate particular registered users on the system.
							\item All information, paper entries and their metadata should be available for viewing to the admin user. Upon selecting a particular user all information associated with that user is displayed.
							\item This action should be logged.
						\end{itemize}
						\item \textbf{Request and Results Data Structures:}
						\begin{figure}[H]
							\includegraphics[width=\linewidth]{../Diagrams/ServiceContracts/User subsystem/SearchAllUsers.jpg}
							\caption{Service Contract - Search All Users}
						\end{figure}
						\begin{figure}[H]
							\includegraphics[width=\linewidth]{../Diagrams/ServiceContracts/User subsystem/ViewAllUserProfiles.jpg}
							\caption{Service Contract - View All User Profiles}
						\end{figure}
					\end{itemize}
					
					\cleardoublepage
					\item \textbf{Purge User from System} \hfill \textit{Priority: Important}
					\begin{itemize}
						\item An admin user must be able to permanently remove a user and all of their associated information from the system.
						\item \textbf{Pre-conditions:}
						\begin{itemize}
							\item The user must be logged in.
							\item The user must have the privileges associated with an admin account.
							\item The user must confirm they want to continue with the action before the system executes the command.
						\end{itemize}
						\item \textbf{Post-conditions:}
						\begin{itemize}
							\item The user and all of their associated information should be removed from the system.
							\item The action should be logged.
						\end{itemize}
						\item \textbf{Request and Results Data Structures:}
						\begin{figure}[H]
							\includegraphics[width=\linewidth]{../Diagrams/ServiceContracts/User subsystem/PurgeUserFromSystem.jpg}
							\caption{Service Contract - Purge User From System}
						\end{figure}
					\end{itemize}
					
					\cleardoublepage
					\item \textbf{Set User as Active/Inactive} \hfill \textit{Priority: Important}
					\begin{itemize}
						\item An admin user should be able to set the status of a user to active or inactive, allowing the user to login or preventing them from continuing to interact with the system.
						\item \textbf{Pre-conditions:}
						\begin{itemize}
							\item The user must be logged in.
							\item The user must have the privileges associated with an admin account.
						\end{itemize}
						\item \textbf{Post-conditions:}
						\begin{itemize}
							\item The selected user should have been marked as active/inactive, active allowing the user to login and use the system, inactive preventing them from login in to the system and making changes.
							\item The action should be logged.
						\end{itemize}
						\item \textbf{Request and Results Data Structures:}
						\begin{figure}[H]
							\includegraphics[width=\linewidth]{../Diagrams/ServiceContracts/User subsystem/SetUserAsActive.jpg}
							\caption{Service Contract - Set User as Active}
						\end{figure}
						\begin{figure}[H]
							\includegraphics[width=\linewidth]{../Diagrams/ServiceContracts/User subsystem/SetUserAsInactive.jpg}
							\caption{Service Contract - Set User as Inactive}
						\end{figure}
					\end{itemize}
					
					\cleardoublepage
					\item \textbf{Add/Remove Users From Any Research Group} \hfill \textit{Priority: Critical}
					\begin{itemize}
						\item An admin user should be able to add or remove a user from any of the research groups within the system.
						\item \textbf{Pre-conditions:}
						\begin{itemize}
							\item The user must be logged in.
							\item The user must have the privileges associated with an admin account.
						\end{itemize}
						\item \textbf{Post-conditions:}
						\begin{itemize}
							\item The selected user should be associated with the specified research group in the case of adding and should be removed and no longer associated with a research group in the case of removing.
							\item The action should be logged.
						\end{itemize}
						\item \textbf{Request and Results Data Structures:}
						\begin{figure}[H]
							\includegraphics[width=\linewidth]{../Diagrams/ServiceContracts/User subsystem/AddUserToAnyGroup.jpg}
							\caption{Service Contract - Add User to Any Group}
						\end{figure}
						\begin{figure}[H]
							\includegraphics[width=\linewidth]{../Diagrams/ServiceContracts/User subsystem/RemoveUserFromAnyGroup.jpg}
							\caption{Service Contract - Remove User from Any Group}
						\end{figure}
					\end{itemize}
				\end{itemize}
				
				\cleardoublepage
			\subsubsection{Notification Sub-System}\label{subsubsec:notification}
				Handles the adding and editing of notifications within the system, as well the process of creating and sending notifications to relevant users of the system.\\
				[3mm]
				\textbf{User:}
				\begin{itemize}
					\item \textbf{Update Paper Deadline} \hfill \textit{Priority: Important}
					\begin{itemize}
						\item Upon creating a paper entry a user must be able to set/update the deadline for when the paper needs to be published by.
						\item \textbf{Pre-conditions:}
						\begin{itemize}
							\item The user must be logged in.
							\item The user must be the author or a co-author of the paper which they would like to set the deadline for.
							\item The paper must not have been terminated.
						\end{itemize}
						\item \textbf{Post-conditions:}
						\begin{itemize}
							\item A deadline will have been set and stored in the system. It will be associated with the particular paper on which it was set.
							\item An update notification should be sent to authors of the paper indicating that the deadline has been set.
							\item The action should be logged.
						\end{itemize}
						\item \textbf{Request and Results Data Structures:}
						\begin{figure}[H]
							\includegraphics[width=\linewidth]{../Diagrams/ServiceContracts/Notification subsystem/UpdatePaperDeadline.jpg}
							\caption{Service Contract - Update Paper Deadline}
						\end{figure}
					\end{itemize}						
				\end{itemize}
				
				\cleardoublepage
				\textbf{\large{Date and Time:}}
				\begin{itemize}
					\item \textbf{Send Deadline Notification} \hfill \textit{Priority: Important}
					\begin{itemize}
						\item The system must be able to notify an author as well as co-authors of a paper of the deadline for the paper.
						\item \textbf{Pre-conditions:}
						\begin{itemize}
							\item A deadline must have been set for a paper.
							\item The deadline must not have passed already.
							\item The authors being notified must be associated with the paper the deadline has been set on.
						\end{itemize}
						\item \textbf{Post-conditions:}
						\begin{itemize}
							\item Authors of a particular paper are notified via the use of email as to what the deadline for the paper is.
							\item The action should be logged.
						\end{itemize}
						\item \textbf{Request and Results Data Structures:}
						\begin{figure}[H]
							\includegraphics[width=\linewidth]{../Diagrams/ServiceContracts/Notification subsystem/SendDeadlineNotification.jpg}
							\caption{Service Contract - Send Deadline Notification}
						\end{figure}
					\end{itemize}
					
					\item \textbf{Send Deadline Update Notification} \hfill \textit{Priority: Important}
					\begin{itemize}
						\item The system must be able to notify an author as well as co-authors of a paper when the deadline for a paper is updated.
						\item \textbf{Pre-conditions:}
						\begin{itemize}
							\item The user must be logged in.
							\item A new deadline must have been set for a paper.
							\item The deadline must not have passed already.
							\item The authors being notified must be associated with the paper the deadline has been updated on.
						\end{itemize}
						\item \textbf{Post-conditions:}
						\begin{itemize}
							\item Authors of a particular paper are notified via the use of email as to what the updated deadline for the paper is.
							\item The action should be logged.
						\end{itemize}
						\item \textbf{Request and Results Data Structures:}
						\begin{figure}[H]
							\includegraphics[width=\linewidth]{../Diagrams/ServiceContracts/Notification subsystem/SendDeadlineUpdateNotification.jpg}
							\caption{Service Contract - Send Deadline Update Notification}
						\end{figure}
					\end{itemize}
				\end{itemize}
				
			\cleardoublepage
			\subsubsection{Reporting Sub-System}\label{subsubsec:report}
				Handles the generation of reports with regards to all relevant information contained by the system.\\
				[3mm]
				\textbf{Head of Research:}
				\begin{itemize}
					\item \textbf{Generate Report for Research Group} \hfill \textit{Priority: Important}
					\begin{itemize}
						\item The head of a research group should be able to generate a summarised report of all important information related to their research group in terms of the users in the group as well as their papers and all metadata associated with their papers.
						\item \textbf{Pre-conditions:}
						\begin{itemize}
							\item The user must be logged in.
							\item The user must have the privileges associated with a Head of Research.
							\item The user must be the head of the research group which they would like to generate a report on.
						\end{itemize}
						\item \textbf{Post-conditions:}
						\begin{itemize}
							\item The head of the research group should have all relevant information returned to him/her in a summarised and easily readable manner.
							\item This action should be logged.
						\end{itemize}
						\item \textbf{Request and Results Data Structures:}
						\begin{figure}[H]
							\includegraphics[width=\linewidth]{../Diagrams/ServiceContracts/Reporting subsystem/GenerateReportForGroup.jpg}
							\caption{Service Contract - Generate Report for Group}
						\end{figure}
					\end{itemize}
				\end{itemize}
				
				\cleardoublepage
				\textbf{Admin:}
				\begin{itemize}
					\item \textbf{Generate Report for Department} \hfill \textit{Priority: Important}
					\begin{itemize}
						\item An admin user should be able to generate a summarised report of all important information related to their department as a whole. This includes summaries of each research group and associated papers, as well as of each individual user of the system and their associated papers.
						\item \textbf{Pre-conditions:}
						\begin{itemize}
							\item The user must be logged in.
							\item The user must have the privileges associated with an admin.
						\end{itemize}
						\item \textbf{Post-conditions:}
						\begin{itemize}
							\item The admin should have all relevant information returned to him/her in a summarised and easily readable manner.
							\item The action should be logged.
						\end{itemize}
						\item \textbf{Request and Results Data Structures:}
						\begin{figure}[H]
							\includegraphics[width=\linewidth]{../Diagrams/ServiceContracts/Reporting subsystem/GenerateReportForDepartment.jpg}
							\caption{Service Contract - Generate Report for Department}
						\end{figure}
					\end{itemize}

					\cleardoublepage
					\item \textbf{Dump Database to File} \hfill \textit{Priority: Nice-to-Have}
					\begin{itemize}
						\item An admin user should be able to at any point dump all of the information within the database into a file for safekeeping, such as a CSV file.
						\item \textbf{Pre-conditions:}
						\begin{itemize}
							\item The user must be logged in.
							\item The user must have the privileges associated with an admin user.
						\end{itemize}
						\item \textbf{Post-conditions:}
						\begin{itemize}
							\item All information contained in the database should be dumped into a CSV file and be made available to download by the admin user.
							\item This action should be logged.
						\end{itemize}
						\item \textbf{Request and Results Data Structures:}
						\begin{figure}[H]
							\includegraphics[width=\linewidth]{../Diagrams/ServiceContracts/Reporting subsystem/DumpDatabaseToFile.jpg}
							\caption{Service Contract - Dump Database to File}
						\end{figure}
					\end{itemize}
				\end{itemize}
			
			\cleardoublepage
			\subsubsection{Group-Control Sub-System}\label{subsubsec:group}
				Handles the creating and removing of research groups, as well as the allocating of heads of research groups and the editing of individual research groups associated metadata.\\
				[3mm]
				\textbf{Head of Research:}
				\begin{itemize}
					\item \textbf{Edit Research Group Information} \hfill \textit{Priority: Critical}
					\begin{itemize}
						\item A head of a research group should be able to edit the information associated with their research group.
						\item \textbf{Pre-conditions:}
						\begin{itemize}
							\item The user must be logged in.
							\item The user must have the privileges associated with a research leader.
							\item The user must be the head of the research group which they are attempting to edit the information of.
						\end{itemize}
						\item \textbf{Post-conditions:}
						\begin{itemize}
							\item The information associated with the particular research group should have been updated within the system.
							\item This action should be logged.
						\end{itemize}
						\item \textbf{Request and Results Data Structures:}
						\begin{figure}[H]
							\includegraphics[width=\linewidth]{../Diagrams/ServiceContracts/Group control subsystem/EditOwnResearchGroup.jpg}
							\caption{Service Contract - Edit Own Research Group}
						\end{figure}
					\end{itemize}
				\end{itemize}
				
				\cleardoublepage
				\textbf{Admin:}
				\begin{itemize}
					\item \textbf{Add Research Group} \hfill \textit{Priority: Critical}
					\begin{itemize}
						\item An admin user should be able to add research groups to the system.
						\item \textbf{Pre-conditions:}
						\begin{itemize}
							\item The user must be logged in.
							\item The user must have the privileges associated with an admin user.
						\end{itemize}
						\item \textbf{Post-conditions:}
						\begin{itemize}
							\item A new research group which is able to have users added to or removed from should exist within the system.
							\item The action should be logged.
						\end{itemize}
						\item \textbf{Request and Results Data Structures:}
						\begin{figure}[H]
							\includegraphics[width=\linewidth]{../Diagrams/ServiceContracts/Group control subsystem/AddResearchGroup.jpg}
							\caption{Service Contract - Add Research Group}
						\end{figure}
					\end{itemize}
					
					\cleardoublepage
					\item \textbf{Set Research Group as Active/Inactive} \hfill \textit{Priority: Important}
					\begin{itemize}
						\item An admin user should be able to set a research group as being active or inactive.
						\item \textbf{Pre-conditions:}
						\begin{itemize}
							\item The user must be logged in.
							\item The user must have the privileges associated with an admin user.
						\end{itemize}
						\item \textbf{Post-conditions:}
						\begin{itemize}
							\item In the case of setting active the research group should be able to have its information edited and have users added to and removed from the group.
							\item In the case of setting inactive the research group should no longer be able to have any operations performed on it.
							\item In both cases the action should be logged.
						\end{itemize}
						\item \textbf{Request and Results Data Structures:}
						\begin{figure}[H]
							\includegraphics[width=\linewidth]{../Diagrams/ServiceContracts/Group control subsystem/SetResearchGroupAsActive.jpg}
							\caption{Service Contract - Set Research Group as Active}
						\end{figure}
						\begin{figure}[H]
							\includegraphics[width=\linewidth]{../Diagrams/ServiceContracts/Group control subsystem/SetResearchGroupAsInactive.jpg}
							\caption{Service Contract - Set Research Group as Inactive}
						\end{figure}
					\end{itemize}

					\cleardoublepage
					\item \textbf{Edit Any Research Group's Information} \hfill \textit{Priority: Critical}
					\begin{itemize}
						\item An admin user should be able to edit the information associated with any of the research groups within the system.
						\item \textbf{Pre-conditions:}
						\begin{itemize}
							\item The user must be logged in.
							\item The user must have the privileges associated with an admin user.
						\end{itemize}
						\item \textbf{Post-conditions:}
						\begin{itemize}
							\item The information associated with the particular research group should have been updated within the system.
							\item This action should be logged.
						\end{itemize}
						\item \textbf{Request and Results Data Structures:}
						\begin{figure}[H]
							\includegraphics[width=\linewidth]{../Diagrams/ServiceContracts/Group control subsystem/EditAnyResearchGroup.jpg}
							\caption{Service Contract - Edit Any Research Group}
						\end{figure}
					\end{itemize}

					\cleardoublepage
					\item \textbf{Allocate/Change Head of Research Group} \hfill \textit{Priority: Critical}
					\begin{itemize}
						\item An admin user should be able to allocate a registered user to be the head of a particular research group, in turn allowing for this user to have more privileges within the system. An admin user should also be able to change the head of a research group.
						\item \textbf{Pre-conditions:}
						\begin{itemize}
							\item The user must be logged in.
							\item The user must have the privileges associated with an admin user.
						\end{itemize}
						\item \textbf{Post-conditions:}
						\begin{itemize}
							\item In the case of allocating the allocated user should now have the necessary rights and privileges to manage and view their allocated research group.
							\item In the case of changing the old head of a research group should have their privileges removed and the new head should be granted those privileges.
							\item This action should be logged.
						\end{itemize}
						\item \textbf{Request and Results Data Structures:}
						\begin{figure}[H]
							\includegraphics[width=\linewidth]{../Diagrams/ServiceContracts/Group control subsystem/AllocateHeadOfResearchGroup.jpg}
							\caption{Service Contract - Allocate Head of Research Group}
						\end{figure}
					\end{itemize}
				\end{itemize}
				
				\cleardoublepage
				\subsubsection{Logging Sub-System}\label{subsubsec:logging}
				Handles the logging of all important interactions which take place on the system.\\
				[3mm]
				\textbf{Admin:}
				\begin{itemize}
					\item \textbf{Download and View Log Text Files} \hfill \textit{Priority: Critical}
					\begin{itemize}
						\item An admin user should be able to download a text file containing the logs of the system.
						\item \textbf{Pre-conditions:}
						\begin{itemize}
							\item The user must be logged in.
							\item The user must have the privileges associated with an admin.
						\end{itemize}
						\item \textbf{Post-conditions:}
						\begin{itemize}
							\item A file should be made available to download containing all the relevant logs which the admin requested.
						\end{itemize}
						\item \textbf{Request and Results Data Structures:}
						\begin{figure}[H]
							\includegraphics[width=\linewidth]{../Diagrams/ServiceContracts/Logging subsystem/DownloadLogFile.jpg}
							\caption{Service Contract - Download Log File}
						\end{figure}
					\end{itemize}
				\end{itemize}
				\textbf{Date and Time:}
				\begin{itemize}
					\item \textbf{Move Current Log File to Backup} \hfill \textit{Priority: Important}
					\begin{itemize}
						\item The system should periodically backup the current log file.
						\item \textbf{Pre-conditions:}
						\begin{itemize}
							\item A certain period of time must have passed.
						\end{itemize}
						\item \textbf{Post-conditions:}
						\begin{itemize}
							\item The old log file should have been backed up.
							\item A new log file should be created.
						\end{itemize}
						\item \textbf{Request and Results Data Structures:}
						\begin{figure}[H]
							\includegraphics[width=\linewidth]{../Diagrams/ServiceContracts/Logging subsystem/ArchiveLog.jpg}
							\caption{Service Contract - Archive Log}
						\end{figure}
					\end{itemize}
					
					\item \textbf{Create New Log File} \hfill \textit{Priority: Important}
					\begin{itemize}
						\item After backing up the previous log file a new one should be created in order to replace the previous one.
						\item \textbf{Pre-conditions:}
						\begin{itemize}
							\item The old log file must have been backed up.
						\end{itemize}
						\item \textbf{Post-conditions:}
						\begin{itemize}
							\item A new log file should replace the older log file so that it can record any future logs.
						\end{itemize}
						\item \textbf{Request and Results Data Structures:}
					\end{itemize}
				\end{itemize}
			
		\cleardoublepage
		\subsection{Required Functionality}\label{subsec:requiredfunctionionality}
			\subsubsection{Publication Sub-System}
			\begin{figure}[H]
				\includegraphics[width=\linewidth]{../Diagrams/Use Cases/Publication subsystem.jpg}
				\caption{Use Case - Publication Sub-System}
			\end{figure}
			
			\cleardoublepage
			\subsubsection{User Sub-System}
			\begin{figure}[H]
				\includegraphics[width=\linewidth]{../Diagrams/Use Cases/User subsystem.jpg}
				\caption{Use Case - User Sub-System}
			\end{figure}
			
			\cleardoublepage
			\subsubsection{Notification Sub-System}
			\begin{figure}[H]
				\includegraphics[width=\linewidth]{../Diagrams/Use Cases/Notification subsystem.jpg}
				\caption{Use Case - Notification Sub-System}
			\end{figure}
			
			\cleardoublepage
			\subsubsection{Reporting Sub-System}
			\begin{figure}[H]
				\includegraphics[width=\linewidth]{../Diagrams/Use Cases/Reporting subsystem.jpg}
				\caption{Use Case - Reporting Sub-System}
			\end{figure}	
			
			\cleardoublepage
			\subsubsection{Group Control Sub-System}
			\begin{figure}[H]
				\includegraphics[width=\linewidth]{../Diagrams/Use Cases/Group control subsystem.jpg}
				\caption{Use Case - Group Control Sub-System}
			\end{figure}
			
			\cleardoublepage
			\subsubsection{Logging Sub-System}
			\begin{figure}[H]
				\includegraphics[width=\linewidth]{../Diagrams/Use Cases/Logging subsystem.jpg}
				\caption{Use Case - Logging Sub-System}
			\end{figure}
			
		\cleardoublepage	

		\subsection{Process Specifications}\label{subsec:processspecification}
		Certain use cases require further information with regards to their function.\\ These use cases are specified further by means of process specification.
		\subsubsection{Publication Sub-System}
			\begin{figure}[H]
				\includegraphics[width=4in, center]{../Diagrams/Process Specifications/Publication subsystem/Add_Edit Own Paper Entry.jpg}
				\caption{User - Add/Edit Own Paper Entry}
			\end{figure}
			\begin{figure}[H]
				\includegraphics[width=4in, center]{../Diagrams/Process Specifications/Publication subsystem/Edit Own Paper Progress.jpg}
				\caption{User - Edit Own Paper Progress}
			\end{figure}
			\begin{figure}[H]
					\includegraphics[width=4in, center]{../Diagrams/Process Specifications/Publication subsystem/Terminate_Resume Own Paper.jpg}
					\caption{User - Terminate/Resume Own Paper}
			\end{figure}
			\begin{figure}[H]
				\includegraphics[width=4in, center]{../Diagrams/Process Specifications/Publication subsystem/Add_Remove Authors from a paper.jpg}
				\caption{User - Add/Remove Authors From a Paper}
			\end{figure}
			\begin{figure}[H]
				\includegraphics[width=4in, center]{../Diagrams/Process Specifications/Publication subsystem/Add_Remove PublicationType From Paper.jpg}
				\caption{User - Add/Remove PublicationType From Paper}
			\end{figure}
			\begin{figure}[H]
				\includegraphics[width=4in, center]{../Diagrams/Process Specifications/Publication subsystem/Add_Edit Author.jpg}
				\caption{User - Add/Edit Author}
			\end{figure}
			\begin{figure}[H]
				\includegraphics[width=4in, center]{../Diagrams/Process Specifications/Publication subsystem/Add_Edit PublicationType.jpg}
				\caption{User - Add/Edit PublicationType}
			\end{figure}
			\begin{figure}[H]
				\includegraphics[width=4in, center]{../Diagrams/Process Specifications/Publication subsystem/Search All Papers.jpg}
				\caption{Head of Research/Admin - Search All Papers}
			\end{figure}
			\begin{figure}[H]
				\includegraphics[width=4in, center]{../Diagrams/Process Specifications/Publication subsystem/Purge Paper.jpg}
				\caption{Admin - Purge Paper}
			\end{figure}	
		\subsubsection{User Sub-System}
			\begin{figure}[H]
				\includegraphics[width=4in, center]{../Diagrams/Process Specifications/User subsystem/Login.jpg}
				\caption{User - Login}
			\end{figure}
			\begin{figure}[H]
				\includegraphics[width=4in, center]{../Diagrams/Process Specifications/User subsystem/Logout.jpg}
				\caption{User - Logout}
			\end{figure}
			\begin{figure}[H]
				\includegraphics[width=4in, center]{../Diagrams/Process Specifications/User subsystem/View Own Profile.jpg}
				\caption{User - View Own Profile}
			\end{figure}
			\begin{figure}[H]
				\includegraphics[width=4in, center]{../Diagrams/Process Specifications/User subsystem/Edit Own Profile.jpg}
				\caption{User - Edit Own Profile}
			\end{figure}
			\begin{figure}[H]
				\includegraphics[width=4in, center]{../Diagrams/Process Specifications/User subsystem/Add_Remove User from any Research Group.jpg}
				\caption{Head of Research - Add/Remove All Users in a Research Group}
			\end{figure}
			\begin{figure}[H]
				\includegraphics[width=4in, center]{../Diagrams/Process Specifications/User subsystem/Add New User.jpg}
				\caption{Admin - Add New User to System}
			\end{figure}
			\begin{figure}[H]
				\includegraphics[width=4in, center]{../Diagrams/Process Specifications/User subsystem/Purge User From System.jpg}
				\caption{Admin - Purge User from System}
			\end{figure}
			\begin{figure}[H]
				\includegraphics[width=4in, center]{../Diagrams/Process Specifications/User subsystem/Set User as Active_Inactive.jpg}
				\caption{Admin - Set User as Active or Inactive}
			\end{figure}
			\begin{figure}[H]
				\includegraphics[width=4in, center]{../Diagrams/Process Specifications/User subsystem/Add_Remove User from any Research Group.jpg}
				\caption{Admin - Add/Remove Users from any Research Group}
			\end{figure}
			\begin{figure}[H]
				\includegraphics[width=4in, center]{../Diagrams/Process Specifications/User subsystem/View All User Profiles.jpg}
				\caption{Admin - Search/View All Users in any Research Group}
			\end{figure}
			
			\subsubsection{Notification Sub-System}
			\begin{figure}[H]
				\includegraphics[width=4in, center]{../Diagrams/Process Specifications/Notification subsystem/Set Paper Deadline.jpg}
				\caption{User - Set Paper Deadline}
			\end{figure}
			\begin{figure}[H]
				\includegraphics[width=4in, center]{../Diagrams/Process Specifications//Notification subsystem/Send Deadline Notification.jpg}
				\caption{System - Send Deadline Notification}
			\end{figure}
			\begin{figure}[H]
				\includegraphics[width=4in, center]{../Diagrams/Process Specifications//Notification subsystem/Send Deadline Update Notification.jpg}
				\caption{System - Send Deadline Update Notification}
			\end{figure}

			\subsubsection{Reporting Sub-System}
			\begin{figure}[H]
				\includegraphics[width=4in, center]{../Diagrams/Process Specifications/Reporting subsystem/Generate Report for Research Group.jpg}
				\caption{Head of Research - Generate Report for Research Group}
			\end{figure}
			\begin{figure}[H]
				\includegraphics[width=4in, center]{../Diagrams/Process Specifications/Reporting subsystem/Generate Report for Department.jpg}
				\caption{Admin - Generate Report for Department}
			\end{figure}
			\begin{figure}[H]
				\includegraphics[width=4in, center]{../Diagrams/Process Specifications/Reporting subsystem/Dump Database to File.jpg}
				\caption{Admin - Dump Database to File}
			\end{figure}
			
			\subsubsection{Group Control Sub-System}
				\begin{figure}[H]
					\includegraphics[width=4in, center]{../Diagrams/Process Specifications/Group control subsystem/Add Research Group.jpg}
					\caption{Admin - Add Research Group}
				\end{figure}
				\begin{figure}[H]
					\includegraphics[width=4in, center]{../Diagrams/Process Specifications/Group control subsystem/Allocate Head of Research Group.jpg}
					\caption{Admin - Allocate Head of Research Group}
				\end{figure}
				\begin{figure}[H]
					\includegraphics[width=4in, center]{../Diagrams/Process Specifications/Group control subsystem/Set Research Group as Active_Inactive.jpg}
					\caption{Admin - Set Research Group as Active/Inactive}
				\end{figure}
				\begin{figure}[H]
					\includegraphics[width=4in, center]{../Diagrams/Process Specifications/Group control subsystem/Edit Research Group Information.jpg}
					\caption{Head of Research/Admin - Edit Research Group Information}
				\end{figure}
			\subsubsection{Logging Sub-System}
				\begin{figure}[H]
					\includegraphics[width=4in, center]{../Diagrams/Process Specifications/Logging subsystem/Archive logs.jpg}
					\caption{DateTime - Archive Logs}
				\end{figure}
				\begin{figure}[H]
					\includegraphics[width=4in, center]{../Diagrams/Process Specifications/Logging subsystem/Download log files.jpg}
					\caption{Admin - Download Log Files}
				\end{figure}
			
		\cleardoublepage
		\subsection{Domain Model}\label{subsec:domainmodel}
		The domain model is described in terms of class diagram.\\ Class diagrams contain information on the current class such as attributes and relationships to other classes.
		
			\begin{figure}[H]				
				\includegraphics[width=\linewidth]{../Diagrams/Domain Model/domainModel.jpg}
				\caption{ Domain model - Classes related to each domain }
			\end{figure}
			\begin{center}
			\textbf{Domains include :}
			\end{center}
		\begin{multicols}{2}
		
			\begin{itemize} 
				\item \textbf{User Domain:}
						\begin{itemize}
						\item User
						\item UserCredentials
						\item Author
						\end{itemize}
					
				\item \textbf{Logging Domain:}
						\begin{itemize}
						\item LogHandler
						\item LogFile
						\end{itemize}				
					
					
				\item \textbf{Group Control Domain:}
						\begin{itemize}
						\item ResearchGroup
						\end{itemize}
						
					\columnbreak

					
				\item \textbf{Reports Domain:}

						\begin{itemize}
						\item ReportGenerator
						\item Report
						\end{itemize}
					
				\item \textbf{Notifications Domain:}
						\begin{itemize}
						\item NotificationHandler
						\item Notification
						\end{itemize}

					
				\item \textbf{Publication Domain:}
						\begin{itemize}
						\item Publication
						\item PublicationType
						\item Conference
						\item Journal
						\end{itemize}		
				
			\end{itemize}
		\end{multicols}
	\cleardoublepage
	\section{Open Issues}\label{sec:issues}
	Issues that weren't discussed in the client requirements meeting:
		\begin{itemize}
		  \item Current representation and layout of system allows for only one department to use the system. Possible problem if system ever needs to expand.
		  \item What technologies and frameworks will we be using moving on in the next phases
		  \item Should each publication have its own super-user, with additional authority over the other users in the paper if said super-user created the paper.
		  \item Should people the administrator requests to aid in their work (e.g. secretaries and assistants) be allowed to receive administrator permissions in order to do certain tasks.
		  \item Should validation of meta data (Non-authentication checks) be handled client side (Ease of use and less hassle for simple validations) or server side (Less chance for SQL injections, JavaScript injections and the like.).
		  \item Will functionality for file storage, version control and such ever possibly be added to this system in the future.
		\end{itemize}
		
\end{document}
