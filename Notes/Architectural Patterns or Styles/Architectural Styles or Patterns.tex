\documentclass[a4paper,12pt]{article}
\begin{document}

\title{Architectural Patterns and Styles - Notes}
\author{Priscilla Madigoe}
\date{08 March 2016}
\maketitle


\section{Architectural Styles}

\subsection{Introduction}
A system typically has various functions to fulfil. These functions can then be divided into subsystems which can be made up of components and connectors, data and control ssections. Architectural styles help decompose a huge system into subsystems effectively. A list of architectural syles will follow below and each one will be discussed in detail. 

\subsection{Types of Architectural Styles}
\begin{itemize}
\item Interacting Processes
\item Dataflow
\item Data-centered
\item Hierarchical
\item Call and Return
\end{itemize}
 
\section{Architectural Patterns}

\subsection{Introduction}
Often times source code needs to be organised so that clear roles, responsibilities and relationships of different modules of the code can be well defined. Architectural patterns help in making this possible and numerous types will be discussed in this section. Using these patterns, code becomes easier to maintain, manage and visualise. They also make understanding of how each component works in a system easier. They are reusable solutions to a commonly occuring problem in Software Architecture within a given context. Patterns generally belong to one of the aforementioned Architectural Styles.

\subsection{Types of Architectural Patterns}

\begin{itemize}
\item Layers Pattern
\item Client-server
\item Representational State Transfer (REST)
\item Master-slave
\item Pipe-filter
\item Broker Pattern
\item Peer-to-peer
\item Event-bus Pattern
\item Model View Controller
\item Blackboard Pattern
\item Interpreter
\end{itemize}


\end{document}