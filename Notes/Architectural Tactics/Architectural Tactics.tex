\documentclass[a4paper,12pt]{article}
\begin{document}

\title{Architectural tactics or strategies - Notes}
\author{Keorapetse Shiko}
\date{09 March 2016}
\maketitle

\cleardoublepage

\section{Architectural Tactics}

	\subsection{Introduction}
	Quality requirements have a significant influence on the software architecture of a system. Architectural tactics are techniques that an 
	architect may use to comprehensively achieve the quality requirements. For each quality requirement listed, potential tactics will be mentioned.

	\subsection{Quality requirements}\label{subsec:quality}
	\begin{itemize}
		\item Performance
		\begin{itemize}
			\item The system is fairly small and simple. Any form or action of enhancing performance may be relatively negligible but a couple of tactics can be implemented. 
			
			\item Tactics
			\begin{itemize}
					\item Spread load across resources
						
						\begin{itemize}
							\item Concurrency can be introduced. Processing different streams of events on different threads and creating additional threads to process different sets of activities can be done by processing requests in parallel. Appropriately allocating threads to resources can maximally exploit concurrency. 
							
						\end{itemize}
					
					\item Resource re-use, caching.
					
						\begin{itemize}
							\item Considering the Readers/Writers problem, multiple readers will have access to view the resource while only one writer can have access to edit the resource. Multiple copies of data can be maintained. Caching is a tactic that replicates data which would be on the same repository to reduce contention.
						\end{itemize}
						
			\end{itemize}		
		\end{itemize}
		\item Reliability
		\begin{itemize}
			
			\item Tactics
			\begin{itemize}
				\item Resource locking
				
					\begin{itemize}
						\item A resource or a part/section of the resource will be locked, only one user will be able to edit that resource. The other users will be notified about the lock by getting conflicts when they try to pull the resource.
					\end{itemize}
										
				\item Maintain backup
				
					\begin{itemize}
						\item The database will have a back up of all of the information that it stores.
					\end{itemize}
								

			\end{itemize}
						
		\end{itemize}
		\item Scalability
		\begin{itemize}
			
			\item Tactics
			\begin{itemize}

				\item Increasing capacity of communication channels
					
					\begin{itemize}
						\item Users will access the system through a website front-end on a web server maintained by the department. The site will be browser independent in order to increase communication channels. The Android system will also be used as a replacement for the website ans vice versa which allow use of mobile phones or tablets.
					\end{itemize}

			\end{itemize}
						
		\end{itemize}
		\item Security
		\begin{itemize}
			
			\item Tactics
			\begin{itemize}
				
				\item Authentication
					
					\begin{itemize}
						\item Users will be authenticated via their usernames and passwords by making use of server side validation.
					\end{itemize}
					
				\item Encryption(hashing passwords)
				
					\begin{itemize}
						 \item Passwords are to be hashed using at least sha256 and should be stored as such within the database along with a salt.
					\end{itemize}				 		
						 
				\item Authorisation
						\begin{itemize}
							\item Users will be grouped by user classes to ensure access control. Users rights to access and modification of data will be determined by the class the user is in.
						\end{itemize}		
					
					
				\item Drop connection
					\begin{itemize}
						\item Sessions will be checked periodically(every 10 minutes). Should a user be inactive for more than the stipulated period, the session will be terminated after the state of the system gets stored for possible retrieval.
					\end{itemize}		
							
				
				
			\end{itemize}			
			
		\end{itemize}
		\item Flexibility
		\begin{itemize}
			\item The client has stated that the system is not needed to be able to extend to accommodate a greater number of departments.
			
			\item Tactics
			\begin{itemize}
				\item Contract based
				\item Dependency injection
			\end{itemize}			
			
		\end{itemize}
		\item Maintainability
		\begin{itemize}
			\item The system should have as few bugs as possible so as to prevent having to constantly maintain it in the future.
			\item The system should be built in a modular way so that all services are decoupled in such a manner that allows for the extension of the system at a later stage.
			
			\item Tactics
			\begin{itemize}
				\item Localise changes(Anticipate expected changes)
				\item Prevention of ripple effect(Use intermediary, maintain existing interface)
				\item Defer binding time(Runtime registration/lookups )
				\item Polymorphism, adherence to defined protocols
			\end{itemize}			
			
		\end{itemize}
		\item Auditability/monitorability
		\begin{itemize}
			\item Every action performed by a user should be logged and all details about said action should be stored.
			\item These actions should be visible to admin users.
			
			\item Tactics
			\begin{itemize}
				\item Logging
			\end{itemize}			
			
		\end{itemize}
		\item Integrability
		\begin{itemize}
			\item User's document entries should not be able to be deleted, if it is a case where the document will not be completed it should remain in the system and be terminated.
			\item A user with no admin rights should not have access to admin privileges so that the system's data may remain integrable and safe.
			\item The system should not ever be in a state where it is under pressure and the data is at risk of becoming corrupted. The system should be designed to handle the pressure for which it has been specified to handle.
			
			\item Tactics
			\begin{itemize}
				\item Publish contracts
				\item Support pluggable adapters
			\end{itemize}			
		
			
		\end{itemize}
		\item Cost
		\begin{itemize}
			\item All software used should not be proprietary but rather open source so as to minimise cost as much as possible.
			
			\item Tactics
			\begin{itemize}
				\item List of open source software to follow..
			\end{itemize}			
			
		\end{itemize}
		\item Usability
		\begin{itemize}
			\item The interface should be lightweight.
			\item The interface should be intuitive to use as well as obey Human Computer Interaction guidelines so that it is efficient and easy to use.
			
			\item Tactics
			\begin{itemize}
				\item Separate user interface
				\item User initiative(Cancel, undo)
				\item System initiative(Maintain user,system and task models)
			\end{itemize}			
			
		\end{itemize}
	\end{itemize}


\end{document}