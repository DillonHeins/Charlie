\documentclass{article}

\usepackage{lipsum}
\usepackage[margin=2.5cm, left=2cm, includefoot]{geometry}

% Header and footer
\usepackage{fancyhdr}
\pagestyle{fancy}

\fancyhead{}
\fancyfoot{}
\fancyfoot[R]{\thepage}
\renewcommand{\headrulewidth}{0pt}
\renewcommand{\footrulewidth}{0pt}
%

\begin{document}

\begin{titlepage}
	\begin{center}
		\line(1,0){300}\\
		[6mm]
		\huge{\bfseries Client Meeting Notes}\\
		[2mm]
		\line(1,0){200}\\
		[15mm]
		\textsc{\large COS301}\\
		[7.5mm]
		\textsc{\large Software Requirements Specification}\\
		[10cm]
	\end{center}
	
	\begin{flushright}
		\textsc{\large Dillon Heins\\
		14035538\\
		16 February 2016\\}
	\end{flushright}
\end{titlepage}

% Table of contents
\tableofcontents
\thispagestyle{empty}
\cleardoublepage
%

% Main body
\pagenumbering{arabic}
\setcounter{page}{1}
%

\section{General Vision}\label{sec:vision}
	\begin{itemize}
		\item A system which allows researchers at the University of Pretoria (specifically within the Computer Science Department) to keep track of the publications which they are currently working on.
		\item This system must also keep track of historical publications so as to allow researchers to maintain a history of their work.
		\item The system should allow the heads of the research groups to manage their group's papers as well as to keep track of all publications under their group.
		\item The head of the Computer Science department, within this system, should be able to view all research groups so as to enable him/her to manage the department's research goals.
	\end{itemize}

\section{Users}\label{sec:users}
	\begin{itemize}
		\item Number of users:
		\begin{itemize}
			\item Members of the department.
			\item Under 100.
		\end{itemize}
		\item System must be able to support all users working at the same time on the system
		\item Only staff members of the University of Pretoria are \textit{users} of the system.
		\begin{itemize}
			\item Staff members act as representatives of student authors when adding them to a paper.
		\end{itemize}
		\item User hierarchy:
		\begin{itemize}
			\item There exists a hierarchy in terms of the privileges which each type of user is granted.
			\item User types:
			\begin{itemize}
				\item Administrative user:
				\begin{itemize}
					\item This user is the HOD and one other possible admin.
					\item Can view all users.
					\item Can add users to system.
					\item Administrative rights.
				\end{itemize}
				\item Head of research group:
				\begin{itemize}
					\item See all research group members' papers.
				\end{itemize}
				\item General user:
				\begin{itemize}
					\item Can only see papers which they have authored themselves or co-authored
					\item Users cannot see other user's publications, only their own
				\end{itemize}
			\end{itemize}
			
		\end{itemize}
	\end{itemize}
	
\section{Document Entries}\label{sec:docs}
	\begin{itemize}
		\item No documents will be stored within the system itself.
		\begin{itemize}
			\item Only the meta-data of documents will be stored.
		\end{itemize}
		\item A document entry consists of:
		\begin{itemize}
			\item Paper title.
		\end{itemize}
	\end{itemize}
	
\end{document}