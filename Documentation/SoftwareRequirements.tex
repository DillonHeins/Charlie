\documentclass{article}

\usepackage[margin=2.5cm,left=2cm,includefoot]{geometry}
\usepackage{graphicx}

% Header and footer
\usepackage{fancyhdr}
\pagestyle{fancy}

\rhead{COS301 - \LaTeX}
\lhead{Team Charlie}
\fancyfoot{}
\fancyfoot[R]{Page \thepage}

\renewcommand{\headrulewidth}{2pt}
\renewcommand{\footrulewidth}{1pt}
%

\begin{document}

	\begin{titlepage}
		\begin{center}
		
			\line(1,0){300}\\
			[6mm]
			\huge{
				\bfseries Software Requirements Specification\\
				and\\
				Technology Neutral Process Design
			}\\
			[2mm]
			\line(1,0){200}\\
			[15mm]
			\textsc{\large (Name of System)}\\
			[7.5mm]
			\textsc{\large University of Pretoria - Team Charlie}\\
			[20mm]
			\textsc{\large Created By:}\\
			[2mm]
			\large{
				Claudio Da Silva - 14205892\\
				Arno Grobler - 14011396\\
				Dillon Heins - 14035538\\
				Charl Jansen Van Vuuren - 13054903\\
				Priscilla Madigoe - 13049128\\
				Bernhard Schuld - 10297902\\
				Keorapetse Shiko - 12231992
			}\\
			[6cm]
		\end{center}
		
		\begin{flushright}
			\textsc{\large 17 February 2016}
		\end{flushright}
	\end{titlepage}
	
	\cleardoublepage
	\thispagestyle{empty}
	\tableofcontents
	
	\cleardoublepage
	\setcounter{page}{1}
	\section{Introduction}\label{sec:intro}
		\subsection{Purpose}\label{subsec:purpose}
			The purpose of this document is to give a detailed explanation and description of the [Name] system. This document will illustrate the purpose as well as the features of said system, the interfaces of the system, the functionality of the system, the constraints under which it must operate and how the system will integrate with external systems. This document has been created for use by the developers of the system, the proposed client as well as any other additional third party collaborators who require to understand the software specifications of the [Name] system.
		
	\cleardoublepage	
	\section{Vision}\label{sec:vision}
		The client has requested a system which allows researchers at the \textit{University of Pretoria}, specifically within the Computer Science Department, to keep track of the publications which they are currently actively involved with or working on.
		
		The system is required to keep track of historical publications so as to allow researches to maintain an archive of their work.
		
		The system should support the management of the multiple research groups within the department as well as allow the acting heads of the individual research groups to manage their group's members and publications.
		
		Ultimately this system is to be used by the acting Head of Department so as to be able to view all the research groups and their research output. It is a way for the department to ensure that the researchers are meeting their goals as well as the department's goals so as to ensure future funding for the department.\\
		[5mm]
		The typical usage scenarios for the desired output from this system would be:
		\begin{itemize}
			\item A UP staff member submitting a research paper to a conference, technical report or conference.
			\item The submission and acceptance of such a paper is what allows researchers to earn units.
			\item These units correspond with academic prestige as well as funding for the University of Pretoria and its researchers.
			\item Departments have predetermined goals which they set out to achieve each academic year.
			\item The ultimate desired output from this system is the ability to monitor the CS Department's researchers and their contribution towards earning these units.
			\item This allows the acting Head of Department to award researchers who achieve as well as take note of those who do not.
			% Desired output is also terminated papers - indicates who is not working
		\end{itemize}
	\cleardoublepage
	\section{Background}\label{sec:background}
		Reasons for the development of this project include but are not limited to:
		\begin{itemize}
			\item Research opportunities:
			\begin{itemize}
				\item Through the monitoring of units earned by staff members it enables the progression of research opportunities for the \textit{University of Pretoria's} Computer Science Department. By monitoring units earned the department is able to ensure that it meets its goals and is able to secure funding opportunities.
			\end{itemize}
			\item Opportunities to simplify some aspect of work:
			\begin{itemize}
				\item The method in use by the department currently is to have all researchers edit the same Microsoft Excel document as their manner of managing and tracking publications and units earned.
				\item This method is inefficient as well as error prone and has hence lead to the need to create this system as a means to replace it.
				\item This system aims to allow for all researchers to be able to manage their own publications in their own user space. It also allows for the Head of Department to no longer have to use an Excel document to create reports from, instead he/she would be able to use the system to the work for him/her in a far more accurate and efficient manner.
			\end{itemize}
			\item Problems the client is currently facing:
			\begin{itemize}
				\item The problem of having all members of a department trying to collaborate on a single Excel document.
				\item The problem of having personal and academic information visible to all who have access to this document.
				\item The problem of managing this data in such a way as to get valuable and meaningful information out of it quickly and accurately.
			\end{itemize}
		\end{itemize}
		
	\cleardoublepage
	\section{Architecture Requirements}\label{sec:architecture}
		\subsection{Access Channel Requirements}\label{subsec:access}
			The different access channels through which the system's services will be made available to users as well as other systems are as follows:
			\begin{itemize}
			  \item An Application Program Interface residing on a server which will be interfaced with by clients in order to supply services to them. Clients referring to:
			  \begin{itemize}
			  	\item Human users via an interface
			  	\item External systems using the services provided by the API
			  \end{itemize}
			  \item Human users can interface with the system via the use of:
			  \begin{itemize}
			  	\item a web-based application service
			  	\item an android based mobile application
			  \end{itemize}
			\end{itemize}
			The interface is required to be lightweight.
		
		\subsection{Quality Requirements}\label{subsec:quality}
			\begin{itemize}
				\item Performance
				\begin{itemize}
					\item Workload is a maximum of 100 users concurrently.
					\item No implementation of concurrent editing of document entries - last saved edit is written to the database.
					\item The system should be able to support 100 users updating information at the same time as updating is more intensive than reading.
					\item The response time of the system should be fast enough that a user is able to complete their work without frustration. Due to the system being off-line it is reasonable to expect the system's response time to be only limited by the speed of the network.
				\end{itemize}
				\item Reliability
				\begin{itemize}
					\item The system should not fail whilst providing critical or important use cases.
					\item The system should not fail at all within a time period of at least 6 months.
				\end{itemize}
				\item Scalability
				\begin{itemize}
					\item Ability for multiple external systems to connect to the system's API.
					\item The system should be able to support a large amount of historical document entries being added to the database.
				\end{itemize}
				\item Security
				\begin{itemize}
					\item A hierarchical system will be used to determine the security privileges of users of the system.
					\item Passwords are to be hashed using at least sha256 and should be stored as such within the database along with a salt.
					\item An inactive user session should be terminated after a period of 10 minutes with no activity.
					\item A user who has forgotten their passwords can use a password reset option which will send a one time password to their registered email address so that they may login once using it and reset their password.
				\end{itemize}
				\item Flexibility
				\begin{itemize}
					\item The client has stated that the system is not needed to be able to extend to accommodate a greater number of departments.
				\end{itemize}
				\item Maintainability
				\begin{itemize}
					\item The system should have as few bugs as possible so as to prevent having to constantly maintain it in the future.
					\item The system should be built in a modular way so that all services are decoupled in such a manner that allows for the extension of the system at a later stage.
				\end{itemize}
				\item Auditability/monitorability
				\begin{itemize}
					\item Every action performed by a user should be logged and all details about said action should be stored.
					\item These actions should be visible to admin users.
				\end{itemize}
				\item Integrability
				\begin{itemize}
					\item User's document entries should not be able to be deleted, if it is a case where the document will not be completed it should remain in the system and be terminated.
					\item A user with no admin rights should not have access to admin privileges so that the system's data may remain integrable and safe.
					\item The system should not ever be in a state where it is under pressure and the data is at risk of becoming corrupted. The system should be designed to handle the pressure for which it has been specified to handle.
				\end{itemize}
				\item Cost
				\begin{itemize}
					\item All software used should not be proprietary but rather open source so as to minimise cost as much as possible.
				\end{itemize}
				\item Usability
				\begin{itemize}
					\item The interface should be lightweight.
					\item The interface should be intuitive to use as well as obey Human Computer Interaction guidelines so that it is efficient and easy to use.
				\end{itemize}
			\end{itemize}
		
		\subsection{Integration Requirements}\label{subsec:integration}
			\begin{itemize}
			  	\item The first item
			  	\item The second item
			  	\item The third etc \ldots
			\end{itemize}
		
		\subsection{Architecture Constraints}\label{subsec:constraints}
			\begin{itemize}
			  \item The first item
			  \item The second item
			  \item The third etc \ldots
			\end{itemize}
		
	\cleardoublepage
	\section{Functional Requirements and Application Design}\label{sec:functional}
		\subsection{Use Case Prioritization}\label{subsec:usecasepriorotization}
			\subsubsection{Paper Sub-System}\label{subsubsec:paper}
				Handles the adding, editing and terminating of papers.\\
				[3mm]
				\textbf{\large{User:}}
				\begin{itemize}
					\item \textbf{Add Paper Entry} \hfill \textit{Priority: Critical}
					\begin{itemize}
						\item A user of the system should be able to add a paper entry associated with their user account into the system.
						\item \textbf{Pre-conditions:}
						\begin{itemize}
							\item A user should have a registered account and be logged in.
						\end{itemize}
						\item \textbf{Post-conditions:}
						\begin{itemize}
							\item A paper entry and all its details should be entered into the database of the system and be associated with a particular registered user.
							\item The act of creating a new paper entry should be logged.
						\end{itemize}
					\end{itemize}
					
					\item \textbf{Edit Own Paper Entry:} \hfill \textit{Priority: Critical}
					\begin{itemize}
						\item A user must be able to edit the metadata associated with their own paper entries.
						\item \textbf{Pre-conditions:}
						\begin{itemize}
							\item A user must be logged in.
							\item The logged in user must be an author or co-author of the paper.
							\item The paper must not have been terminated.
						\end{itemize}
						\item \textbf{Post-conditions:}
						\begin{itemize}
							\item The metadata in the database associated with the paper should be updated.
							\item The action of editing and all relevant changes to the entry must be logged.
						\end{itemize}
					\end{itemize}
					
					\item \textbf{Add/Remove Authors from a Paper:} \hfill \textit{Priority: Critical}
					\begin{itemize}
						\item A user should be able to add and remove authors from a paper which they have created or are a co-author of.
						\item \textbf{Pre-conditions:}
						\begin{itemize}
							\item The user must be logged in.
							\item The user must be the author or a co-author of the paper.
						\end{itemize}
						\item \textbf{Post-conditions:}
						\begin{itemize}
							\item There should be an ordered list of unlimited size associated with the paper consisting of the details of the author and co-authors stored in the database.
							\item The action should be logged.
						\end{itemize}
					\end{itemize}
					
					\item \textbf{Add New Author:} \hfill \textit{Priority: Critical}
					\begin{itemize}
						\item A user should have the option to add the details of a new author into the system if the author does not already exist within the system.
						\item \textbf{Pre-conditions:}
						\begin{itemize}
							\item The user must be logged in.
							\item The author must not already exist. 
						\end{itemize}
						\item \textbf{Post-conditions:}
						\begin{itemize}
							\item The author and his/her associated details have been added to the system's database.
							\item The action should be logged.
						\end{itemize}
					\end{itemize}					
					
					\item \textbf{Search Authors from Database:} \hfill \textit{Priority: Critical}
					\begin{itemize}
						\item The user should be able to search or select from a list of authors within the database to associate the authors' entries with their paper.
						\item \textbf{Pre-conditions:}
						\begin{itemize}
							\item The user is logged in.
							\item The author is within the system.
							\item The author is not within the system.
						\end{itemize}
						\item \textbf{Post-conditions:}
						\begin{itemize}
							\item If the author is within the system a result matching the search term or the selected item should be returned.
							\item If the author is not within the system no result should be returned and it should be indicated as such. An option to add this author to the system's database should be offered to the user.
							\item The action should be logged.
						\end{itemize}
					\end{itemize}
					
					\item \textbf{Edit Own Paper Progress:} \hfill \textit{Priority: Important}
					\begin{itemize}
						\item A user should have the ability to indicate in terms of a percentage how far they are currently in terms of progress on their paper.
						\item \textbf{Pre-conditions:}
						\begin{itemize}
							\item A user should be logged in.
							\item The logged in user must be an author or co-author of the paper.
							\item The paper must not have been terminated.
						\end{itemize}
						\item \textbf{Post-conditions:}
						\begin{itemize}
							\item The field indicating progress within the database must have been updated.
							\item The change in progress should be depicted visually to the user.
							\item The action should be logged.
						\end{itemize}
					\end{itemize}
					
					\item \textbf{Terminate Own Paper:} \hfill \textit{Priority: Critical}
					\begin{itemize}
						\item A user must be able to terminate (make inactive) a paper which they feel they are not going to be completing in the future.
						\item \textbf{Pre-conditions:}
						\begin{itemize}
							\item A user must be logged in.
							\item The logged in user must be an author or co-author of the paper.
						\end{itemize}
						\item \textbf{Post-conditions:}
						\begin{itemize}
							\item The paper has been marked as terminated within the database.
							\item The action of terminating the paper must have been logged.
							\item The paper must no longer be able to be edited by the user.
						\end{itemize}
					\end{itemize}
					
					\item \textbf{Add/Remove Conference/Journal to Paper:} \hfill \textit{Priority: Critical}
					\begin{itemize}
						\item Whilst creating or editing a paper a user should be able to associate a particular conference or journal with their paper as well as be able to remove an already associated conference or journal whilst editing a paper.
						\item \textbf{Pre-conditions:}
						\begin{itemize}
							\item The user must be logged in.
							\item The user must be an author or co-author of the paper in question.
						\end{itemize}
						\item \textbf{Post-conditions:}
						\begin{itemize}
							\item The paper should be associated with a particular conference or journal or the paper should have had its association removed.
							\item The action should be logged.
						\end{itemize}
					\end{itemize}					
					
					\item \textbf{Add New Conference/Journal:} \hfill \textit{Priority: Critical}
					\begin{itemize}
						\item A user must be able to fill in the details of a particular conference or journal if it does not already exist within the system.
						\item \textbf{Pre-conditions:}
						\begin{itemize}
							\item The user must be logged in.
							\item The conference or journal must not already exist.
						\end{itemize}
						\item \textbf{Post-conditions:}
						\begin{itemize}
							\item The conference or journal and all its details have been added to the system's database.
							\item The action has been logged.
						\end{itemize}
					\end{itemize}					
					
					\item \textbf{Search Conferences/Journals in Database:} \hfill \textit{Priority: Critical}
					\begin{itemize}
						\item The user should be able to search or select an already existing conference/journal from a list populated through the searching of the database.
						\item \textbf{Pre-conditions:}
						\begin{itemize}
							\item The user must be logged in.
							\item The conference/journal is within the system.
							\item The conference/journal is not within the system.
						\end{itemize}
						\item \textbf{Post-conditions:}
						\begin{itemize}
							\item If the conference/journal is within the system a result matching the search term or the selected item should be returned.
							\item If the conference/journal is not within the system no result should be returned and it should be indicated as such. An option to add this conference/journal into the system should also be offered.
							\item The action should be logged.
						\end{itemize}
					\end{itemize}					
				\end{itemize}
				\textbf{\large{Head of Research:}}
				
				\begin{itemize}
					\item \textbf{Search All Papers within Research Group} \hfill \textit{Priority: Critical}
					\begin{itemize}
						\item The head of a research group should have access to all the papers associated with their particular research group.
						\item \textbf{Pre-conditions:}
						\begin{itemize}
							\item The user must be logged in.
							\item The user must have the privilege rights assigned to a head of research.
							\item The user must be the head of the particular research group they would like to observe.
						\end{itemize}
						\item \textbf{Post-conditions:}
						\begin{itemize}
							\item The user should be able to view all metadata associated with the papers of their research group.
							\item The action should be logged.
						\end{itemize}
					\end{itemize}
				\end{itemize}
				\textbf{\large{Admin:}}
				
				\begin{itemize}
					\item \textbf{Search All Papers on System} \hfill \textit{Priority: Critical}
					\begin{itemize}
						\item An admin user should be able to view all papers and their metadata on the system.
						\item \textbf{Pre-conditions:}
						\begin{itemize}
							\item The user must be logged in.
							\item The user must have the privilege rights assigned to an admin.
						\end{itemize}
						\item \textbf{Post-conditions:}
						\begin{itemize}
							\item The user should be able to view all metadata associated with all the papers stored on the system.
							\item The action should be logged.
						\end{itemize}
					\end{itemize}
					
					\item \textbf{Purge Paper From the System} \hfill \textit{Priority: Important}
					\begin{itemize}
						\item An admin user should be able to completely remove a paper from the system if they find the need to do so.
						\item \textbf{Pre-conditions:}
						\begin{itemize}
							\item The user must be logged in.
							\item The user must have the privilege rights assigned to an admin.
						\end{itemize}
						\item \textbf{Post-conditions:}
						\begin{itemize}
							\item The paper and all of its metadata should be permanently removed from the system.
							\item The action should be logged.
						\end{itemize}
					\end{itemize}
				\end{itemize}
			\subsubsection{User Sub-System}\label{subsubsec:user}
				Handles all actions associated with user accounts such as login, viewing and editing profiles, as well as privileged actions such as adding/removing users to the system and adding/removing users from a research group.\\
				[3mm]
				\textbf{\large{User:}}
				\begin{itemize}
					\item \textbf{Login} \hfill \textit{Priority: Critical}
					\begin{itemize}
						\item A registered user of the system must be able to login to the system using their private credentials.
						\item \textbf{Pre-conditions:}
						\begin{itemize}
							\item The user must not be logged in.
							\item The identifying field used (email address) must match one of an active, registered user of the system.
							\item The password entered must match the password associated with the identifying field within the system.
						\end{itemize}
						\item \textbf{Post-conditions:}
						\begin{itemize}
							\item If the credentials were incorrect a notification must be displayed, as well as an option to reset passwords.
							\item If the credentials were correct:
							\begin{itemize}
								\item The system must begin a session for the user.
								\item The user must gain access to the system according to the set of privileges associated with their account.
							\end{itemize}														
							\item In both cases the action should be logged.
						\end{itemize}
					\end{itemize}
					
					\item \textbf{View Own User Profile} \hfill \textit{Priority: Critical}
					\begin{itemize}
						\item A logged in user should be able to see their own profile and all details associated with it. The user should also be able to view all papers associated with their account.
						\item \textbf{Pre-conditions:}
						\begin{itemize}
							\item The user must be logged in.
							\item The user's profile being viewed must be the same as the logged in user.
						\end{itemize}
						\item \textbf{Post-conditions:}
						\begin{itemize}
							\item The information associated with the user's account should be displayed to them.
							\item The action should be logged.
						\end{itemize}
					\end{itemize}					
				\end{itemize}
				\textbf{\large{Head of Research:}}
				
				\begin{itemize}
					\item \textbf{Add/Remove Users from Research Group} \hfill \textit{Priority: Critical}
					\begin{itemize}
						\item The head of a research group should be able to manage the users associated with that group.
						\item \textbf{Pre-conditions:}
						\begin{itemize}
							\item The user must be logged in.
							\item The user must have the privileges associated with a Head of Research.
							\item The user must be the Head of Research of the particular group which they are wanting to manage.
						\end{itemize}
						\item \textbf{Post-conditions:}
						\begin{itemize}
							\item The user must have been able to add other registered users into their research group.
							\item The user must have been able to remove other registered users from their research group.
							\item The action should be logged.
						\end{itemize}
					\end{itemize}
					
					\item \textbf{Search All Users in Research Group} \hfill \textit{Priority: Critical}
					\begin{itemize}
						\item The head of a research group should be able to search all of the users within their research group.
						\item \textbf{Pre-conditions:}
						\begin{itemize}
							\item The user must be logged in.
							\item The user must have the privileges associated with a Head of Research.
							\item The user must be the Head of Research of the particular group which they are wanting to search.
						\end{itemize}
						\item \textbf{Post-conditions:}
						\begin{itemize}
							\item The user must have been able to view all of the users associated with their research group.
							\item The action should be logged.
						\end{itemize}
					\end{itemize}
					
					\item \textbf{View All User Profiles in Research Group} \hfill \textit{Priority: Critical}
					\begin{itemize}
						\item A research group leader should be able to access the profiles and information associated with the users within their research group. In particular the Head of Research should be able to view all paper entries associated with each user.
						\item \textbf{Pre-conditions:}
						\begin{itemize}
							\item The user must be logged in.
							\item The user must have the privileges associated with a Head of Research.
							\item The user must be the Head of Research of the particular group which they are wanting to view profiles of.
						\end{itemize}
						\item \textbf{Post-conditions:}
						\begin{itemize}
							\item The head of the research group can access and view all profile information (including paper entries) for each of the users associated with their research group.
							\item The action should be logged.
						\end{itemize}
					\end{itemize}					
				\end{itemize}
				\textbf{\large{Admin:}}
				
				\begin{itemize}
					\item \textbf{Add New User} \hfill \textit{Priority: Critical}
					\begin{itemize}
						\item Admin users should have the ability to add new users into the system.
						\item \textbf{Pre-conditions:}
						\begin{itemize}
							\item The user must be logged in.
							\item The user must have the privileges associated with an admin account.
						\end{itemize}
						\item \textbf{Post-conditions:}
						\begin{itemize}
							\item A new user with all their metadata as well as privileges should be added into the system.
							\item The action should be logged.
						\end{itemize}
					\end{itemize}
					
					\item \textbf{Search All Users} \hfill \textit{Priority: Critical}
					\begin{itemize}
						\item Admin users should be able to search through all of the users within the system.
						\item \textbf{Pre-conditions:}
						\begin{itemize}
							\item The user must be logged in.
							\item The user must have the privileges associated with an admin account.
						\end{itemize}
						\item \textbf{Post-conditions:}
						\begin{itemize}
							\item The admin user should have been able to search and locate particular registered users on the system.
							\item This action should be logged.
						\end{itemize}
					\end{itemize}
					
					\item \textbf{Purge User from System} \hfill \textit{Priority: Nice-to-Have}
					\begin{itemize}
						\item An admin user must be able to permanently remove a user and all of their associated information from the system.
						\item \textbf{Pre-conditions:}
						\begin{itemize}
							\item The user must be logged in.
							\item The user must have the privileges associated with an admin account.
							\item The user must confirm they want to continue with the action before the system executes the command.
						\end{itemize}
						\item \textbf{Post-conditions:}
						\begin{itemize}
							\item The user and all of their associated information should be removed from the system.
							\item The action should be logged.
						\end{itemize}
					\end{itemize}
					
					\item \textbf{Set User as Active/Inactive} \hfill \textit{Priority: Nice-to-Have}
					\begin{itemize}
						\item An admin user should be able to set the status of a user to active or inactive, allowing the user to login or preventing them from continuing to interact with the system.
						\item \textbf{Pre-conditions:}
						\begin{itemize}
							\item The user must be logged in.
							\item The user must have the privileges associated with an admin account.
						\end{itemize}
						\item \textbf{Post-conditions:}
						\begin{itemize}
							\item The selected user should have been marked as active/inactive, active allowing the user to login and use the system, inactive preventing them from login in to the system and making changes.
							\item The action should be logged.
						\end{itemize}
					\end{itemize}					
					
					\item \textbf{View All User Profiles} \hfill \textit{Priority: Critical}
					\begin{itemize}
						\item An admin user should be able to view all the information associated with each of the registered users within a system, especially their paper entries and all of the metadata associated with them.
						\item \textbf{Pre-conditions:}
						\begin{itemize}
							\item The user must be logged in.
							\item The user must have the privileges associated with an admin account.
						\end{itemize}
						\item \textbf{Post-conditions:}
						\begin{itemize}
							\item All information, paper entries and their metadata should be available for viewing to the admin user. Upon selecting a particular user all information associated with that user is displayed.
							\item The action should be logged.
						\end{itemize}
					\end{itemize}
				\end{itemize}
				
			
		\subsection{Use Case/ Service Contracts}\label{subsec:servicecontracts}
		Each use case is discussed in detail in this section
		%We can use this formatting per use case
			\subsubsection{"use case name"}
			\begin{description}
			\item[Description:] "description of the use case"
				\item[Pre-conditions:]
			  	\begin{itemize}
			  		\item 
			  		\item
			  	\end{itemize}
			  \item[Post-conditions:] 
			  	\begin{itemize}
			  		\item 
			  		\item
			  	\end{itemize}
			  \item[Request and Results Data structures:]
			   \begin{itemize}
			  		\item 
			  		\item
			  	\end{itemize}
			\end{description}
			
		\subsection{Required Functionality}\label{subsec:requiredfunctionionality}
			%\begin{figure}[h]
			%\includegraphics[width=4in]{imagename}
			%\caption{Figure information}
			%\end
	
		\subsection{Process Specifications}\label{subsec:processspecification}
		Certain use cases require further information with regards to their function.\\ These use cases are specified further by means of process specification.
			%\begin{figure}[h]
			%\includegraphics[width=4in]{imagename}
			%\caption{Figure information}
			%\end
		
		\subsection{Domain Model}\label{subsec:domainmodel}
		The domain model is described in terms of class diagram.\\ Class diagrams contain information on the current class such as attributes and relationships to other classes.
			%\begin{figure}[h]
			%\includegraphics[width=4in]{imagename}
			%\caption{Figure information}
			%\end
		
	\cleardoublepage
	\section{Open Issues}\label{sec:issues}
		\begin{itemize}
		  \item The first item
		  \item The second item
		  \item The third etc \ldots
		\end{itemize}
		
\end{document}
