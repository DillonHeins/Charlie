\documentclass{article}

\usepackage{lipsum}
\usepackage[margin=2.5cm, left=2cm, includefoot]{geometry}

% Header and footer
\usepackage{fancyhdr}
\pagestyle{fancy}

\fancyhead{}
\fancyfoot{}
\fancyfoot[R]{\thepage}
\renewcommand{\headrulewidth}{0pt}
\renewcommand{\footrulewidth}{0pt}
%

\begin{document}

\begin{titlepage}
	\begin{center}
		\line(1,0){300}\\
		[6mm]
		\huge{\bfseries Client Meeting Notes}\\
		[2mm]
		\line(1,0){200}\\
		[15mm]
		\textsc{\large COS301}\\
		[7.5mm]
		\textsc{\large Software Requirements Specification}\\
		[10cm]
	\end{center}
	
	\begin{flushright}
		\textsc{\large Dillon Heins\\
		14035538\\
		16 February 2016\\}
	\end{flushright}
\end{titlepage}

% Table of contents
\tableofcontents
\thispagestyle{empty}
\cleardoublepage
%

% Main body
\pagenumbering{arabic}
\setcounter{page}{1}
%

\section{General Vision}\label{sec:vision}
	\begin{itemize}
		\item A system which allows researchers at the University of Pretoria (specifically within the Computer Science Department) to keep track of the publications which they are currently working on.
		\item This system must also keep track of historical publications so as to allow researchers to maintain a history of their work.
		\item The system should allow the heads of the research groups to manage their group's papers as well as to keep track of all publications under their group.
		\item The head of the Computer Science department, within this system, should be able to view all research groups so as to enable him/her to manage the department's research goals.
	\end{itemize}

\section{Users}\label{sec:users}
	\begin{itemize}
		\item Number of users:
		\begin{itemize}
			\item members of the department
			\item under 100
		\end{itemize}
		\item The system must be able to support all users working at the same time on the system.
		\item Users can see a profile page of another user but this profile page contains very little information:
		\begin{itemize}
			\item name and surname
			\item email
			\item staff number (if any)
			\item other contact details
		\end{itemize}
		\item Only staff members of the University of Pretoria are \textit{users} of the system.
		\begin{itemize}
			\item Staff members act as representatives of student authors when adding them to a paper.
		\end{itemize}
		\item Users must be able to edit their user information.
		\item A user must be able to assign roles to authors of their paper.
		\item User hierarchy:
		\begin{itemize}
			\item There exists a hierarchy in terms of the privileges which each type of user is granted.
			\item User types:
			\begin{itemize}
				\item Administrative user:
				\begin{itemize}
					\item this user is the HOD and one other possible admin
					\item can view all users
					\item can add users to system
					\item administrative rights
				\end{itemize}
				\item Head of research group:
				\begin{itemize}
					\item see all research group members' papers
					\item can possibly add papers on behalf of others
				\end{itemize}
				\item General user:
				\begin{itemize}
					\item can only see papers which they have authored themselves or co-authored
					\item users cannot see other user's publications, only their own
				\end{itemize}
			\end{itemize}
		\end{itemize}
	\end{itemize}

\section{Document Entries}\label{sec:docs}
	\begin{itemize}
		\item No documents will be stored within the system itself.
		\begin{itemize}
			\item Only the meta-data of documents will be stored.
		\end{itemize}
		\item A document entry consists of:
		\begin{itemize}
			\item paper title
			\item authors
			\begin{itemize}
				\item author list must be ordered by priority
				\item number of authors per paper is unlimited
			\end{itemize}
			\item publication type/intended venue
			\begin{itemize}
				\item conference
				\item journal paper
				\item technical report
				\item accredited
			\end{itemize}
			\item publication target
			\item dates (deadlines)
			\item progress
			\item indication of whether it has been submitted or not
			\item indication of whether it is waiting for review
		\end{itemize}
		\item A document entry must have at least one author.
		\item Papers can be published across multiple research groups.
		\item Deadlines for papers are optional when a user creates a paper.
			\begin{itemize}
				\item The system can ask the user to enter a deadline if the publication type is a conference.
			\end{itemize}
		\item If paper's intended venue is a special editions journal article the system should add in the deadline for the user.
		\item It should be possible to enter historical entries.
		\item Rejected papers must remain in a user's personal history.
		\begin{itemize}
			\item The number of tries per paper (different venues) must be kept track of.
		\end{itemize}
		\item Paper's cannot be removed from the system, they can only be edited or if they are no longer active they can be terminated.
	\end{itemize}
	
\cleardoublepage
\section{Non-users}\label{sec:non-users}
	\begin{itemize}
		\item A non-user is an author of a paper who is not a member of staff of the Computer Science department.
		\item Non-user types:
		\begin{itemize}
			\item student of the University of Pretoria
			\item an external collaborator
		\end{itemize}
		\item A user must be able to specify whether a non-user is a student of the University of Pretoria or whether they are an external collaborator.
		\item A non-user's details must still be saved within the system however there will be no login account associated with a non-user.
		\item Previously entered non-user authors should be displayed as options when selecting authors for a paper.
		\item Non-user details must be able to be edited by a user of the system.
		\item There must be functionality to allow an admin to assign a non-user as a user of the system.
	\end{itemize}
	
\section{Units}\label{sec:units}
	\begin{itemize}
		\item Intended venues have different unit ratings.
		\item If an intended venue is accredited then it has a different weighting in terms of units.
		\item The units earned by each user must be kept track of.
		\item The default units earned by each intended venue is able to be set and edited by the admin.
	\end{itemize}
	
\section{Architecture Requirements}\label{sec:architecture}
	\begin{itemize}
		\item Access channel requirements:
		\begin{itemize}
			\item A central application program interface will be run on a server and clients will interface with it to access services from it via:
			\begin{itemize}
				\item an android application
				\item a web application interface
			\end{itemize}
			\item The interface must be lightweight.
		\end{itemize}
		\item Integration requirements:
		\begin{itemize}
			\item The system might have to be able to eventually integrate with external systems.
		\end{itemize}
		\item Quality requirements:
		\begin{itemize}
			\item Performance: No concurrent editing of documents necessary, the last submitted edit is what will be saved to the database.
			\item Auditability: All actions performed by all user must be logged and stored in detail.
			\item Integrity: No need for backups through system.
		\end{itemize}
	\end{itemize}

\section{System}\label{sec:system}
	\begin{itemize}
		\item The system should be able to email users within the system with regards to reminders for deadlines.
	\end{itemize}
\end{document}